\chapter{Introduction}
\label{chap:intro}
%

The complete hereditary information for humans is contained in the approximately three billion (\begin{math}3\times10^9\end{math})
deoxyribonucleic acid (DNA) basepairs of the human genome, which provides a blueprint for managing all biological processes
required by the human body~\citep{Lander2001}.  The best understood class of elements in the genome are the 20,000 coding genes, which provide instructions
to produce the proteins that comprise much of the human body~\citep{Clamp2007}.  Despite being the origin of the name ``genome'', the 20,000 protein
coding ``genes'' comprise only 1.5\% of the basepairs in the genome~\citep{Hsu2006}.
%

All of the approximately ten trillion (\begin{math}10^{13}\end{math}) cells in an adult human derive from a series of duplications of
a single zygote cell~\citep{Alberts2002}.  Thus (with a few notable exceptions), all cells contain the same genome and hence the same instructions.  Yet,
because different cells utilize different portions of the shared genome, the human body contains hundreds of different cell types, each with
markedly distinct properties~\citep{Alberts2002}.
%

Approximately one million (\begin{math}10^6\end{math}) \cis-regulatory elements in the genome form
a control layer and function as switches that activate or inactivate genes in specific cell types~\citep{Maston2006}.  By current estimates, \cis-regulatory
elements comprise up to 10\% of the genome~\citep{Garber2009}, meaning that much more of the genome is devoted to controlling gene
activity than to encoding how to produce proteins.  The grammar of this large and critical class of genomic elements is poorly understood.
In other words, current knowledge is too limited to effectively predict the function of a \cis-regulatory element from its DNA sequence.
A fuller understanding of \cis-regulatory grammar would empower studies of development, evolution, and disease~\citep{Davidson2006}.
%

\section{Organization}
%

In this work, I describe my efforts to annotate the biological function of the \cis-regulatory elements in the human genome and further
to utilize an improved understanding of the grammar of \cis-regulatory elements to uncover gene regulatory networks that underlie
specific biological processes.
%

\chapref{chap:cisreg} introduces the classes of \cis-regulatory elements and transcription factors,
describes methods for characterizing the biophysical binding preferences of specific transcription factors for DNA, and details the 
GREAT approach~\citep{McLean2010} for inferring the biological function of a set of regulatory elements.
%

\chapref{chap:prism} introduces two novel methods: excess conservation based binding site prediction and the PRISM approach to
annotating gene regulatory networks.  I develop the novel excess conservation score and demonstrate that it locates transcription factor
binding sites in the genome more accurately than the previous state of the art.  I combine the improved binding site prediction with
GREAT in a novel statistical framework that infers gene regulatory networks through which a transcription factor regulates
target genes in a specific biological process.  The combination of accurate binding site prediction and GREAT is termed PRISM for
``Predicting Regulatory Information from Single Motifs''.  The PRISM approach identifies 2,543 transcription factor function predictions
that annotate 109,748 mostly distal binding sites near 7,692 different target genes across the human and mouse genomes.
%

As complement to the PRISM method's broad screen of regulatory function in the human genome, \chapref{chap:neocortex} details an application
of the same tools to a focused case study of neocortex development.  I introduce a novel approach that combines computational
analysis of regulatory elements with high-throughput sequencing techniques to uncover gene interaction networks in a rapid, non-targeted
manner that is not possible with classical genetics methods.
%

\chapref{chap:conclusion} discusses conclusions and suggestions for future work.
%
