\chapter{Introduction}
\label{chap:intro}

The human genome consists of 3 billion A, C, G, and T deoxyribonucleic acid (DNA) bases that provide the instructions for making every cell in our bodies. The central dogma of biology tells us that these instructions, via ribonucleic acid (RNA) intermediates, encode the information necessary for synthesizing proteins, which in turn act as the cellular machinery of the cell. Over the past decade, the plummeting cost of DNA sequencing has allowed scientists to measure the identity of these 3 billion nucleotides, revealing that protein coding sequence only occupies ~2\% of the human genome.

The precipitous drop in the cost of sequencing has also driven scientists to dream up new methods for surveying the entirety of what is going on within the cell. In addition to measuring the identity of the 3 billion nucleotides that make up the human genome, DNA sequencing can now be used to measure variation in these nucleotides between individuals, protein binding and chemical modifications to these nucleotides, and the repetoire of genes being expressed within cells. Functional sequencing of transcription factor binding sites, epigenomic marks, open chromatin, and more, as well as the sequencing of other species and cross-species conservation, suggest that as much as 10 times more genomic sequence (20\% of the genome) is dedicated to regions that control gene expression. In contrast with the amino acid code, which specifies a strict grammar for translating 3-nucleotide codons into the amino acid sequences that make up proteins, the underlying logic of these regulatory sequences remains poorly understood.

The central question of genomics is understanding the instructions encoded within our DNA - going from genotype to phenotype. From a medical perspective, the consequences of these instructions, particularly when they undergo a change due to a mutation, are of particular interest. Nature's own experiments, naturally occurring mutations resulting in disease, coupled with finding similar mutations causing the same disease in other individuals or filtering against the background of mutations found in thousands of healthy individuals has begun to reveal the specific changes to the instructions encoded within our DNA that cause disease. As the price of sequencing continues to drop, we reveal more and more of the genomic underpinning of human diseases and also highlight sequences for biological significance.

\section{Gene regulation}

\subsection{Transcriptional regulation}

The 2\% of the human genome dedicated to protein coding sequence is divided into approximately 20,000 genes, but these are controlled by approximately 1,000,000 gene regulatory sequences. These regulatory sequences control when, where, and how much of each gene is produced in different cell types at different times{Levo:2014hl}. A typical gene regulatory enhancer is over 100 base pairs (bp) long and encodes a landing pad for multiple transcription factors to bind{Spitz:2012ir}. Through enhancer-promoter looping, these transcription factors recruit RNA polymerase II and result in the transcription of the enhancer's target gene in the appropriate spatio-temporal pattern.

\subsection{Fan in}

The mammalian genome contains multiple large gene deserts carrying numerous conserved and likely \emph{cis}-regulatory sequences{Ovcharenko:2005br}. Experiments in yeast have revealed that gene expression monotonically increases (and eventually saturates) as the number of tandem transcription factor binding sites in the promotor of that gene increases{Sharon:2012io}. The purpose of multiple distinct regulatory sequences next to the same gene, however, remains unknown. It is possible that multiple enhancers buffer against environmental variability, act in an additive fashion to create the complete spatio-temporal expression pattern of a gene, or fine-tune the precise levels of gene expression. Providing a further complication for functional interpretation of multiple enhancers of the same gene, while cross-species conservation may reveal likely \emph{cis}-regulatory sequences, it is impossible to deduce from sequence patterns alone how many \emph{cis}-regulatory regions are active simultaneously in any given biological context.

The sequencing of the DNA bound by transcriptional co-activators or labeled by certain epigenomic marks are effective methods for annotating active enhancers in specific biological contexts genome-wide{Visel:2009jp, Ernst:2011kw}. For example, over 80\% of genomic regions marked by EP300, a transcriptional co-activator, in embryonic day 11.5 (E11.5) mouse forebrain showed enhancer activity when tested in vivo{Visel:2009jp}. In previous work, we measured the active enhancer landscape of E14.5 dorsal cerebral cortex by harvesting tissue from mouse embryos, isolating chromatin, and performing chromatin immunoprecipitation with massively parallel DNA sequencing (ChIP-seq) for EP300. Multiple genes were flanked by dozens of candidate enhancers each, suggesting the high “fan in” of a complex regulatory program. These heavily regulated genes include key neocortical genes, as well as suspected and novel genes{Wenger:2013jd}, suggesting that those genes with the highest "fan in" represent important genes within a given biological context.

\subsection{Fan out}

While a single gene may have many adjacent enhancers, the converse is also true: transcription factors may regulate many different target genes. From the perspective of the regulators in the network, ChIP-seq of different transcription factors has repeatedly shown that a given transcription factor binds near hundreds or even thousands of genes specific to the contexts it is known to regulate, suggesting a high “fan out” of transcription regulation{McLean:2010iq}. This means that a given transcription factor can simultaneously activate entire pathways of downstream genes. By collectively examining the annotated functions of the target genes downstream of a given transcription factor, we can gain insights into its biological roles{McLean:2010iq, Wenger:2013ds}. Distinguishing transcription factor binding from transcription factor binding that causes a change in gene expression from transcription factor binding that causes an observable phenotype or loss of fitness when ablated is essential, however, for understanding the regulatory role of a transcription factor with respect to specific target genes.

\section{Identification of disease-causing mutations}

\subsection{Exome sequencing}

Protein coding DNA makes up just 2\% of the genome, but a disproportionate amount is known about these sequences relative to those comprising regulatory enhancers. Mutations resulting in changes to the amino acid sequence of a protein can have dramatic effects, which is reflected in strong cross-species sequence conservation. In contrast, due to the degeneracy of transcription factor binding motifs and regulatory redundancy, mutational changes to regulatory sequences may only result in a small phenotypic effect. By providing a smaller target, being well annotated and understood, and consisting of DNA that is generally highly intolerant to change, protein coding sequences are an attractive target for finding mutations causal for different diseases.

The sequencing of just the protein coding portions of the genome, or exome sequencing, allows researchers to probe this fraction of the genome at a much lower cost. By sequencing the exomes of families having a disease and identifying mutations that segregate with the disease, researchers can identify the causal mutation. Frequently, however, DNA samples of an entire family are not available, or a sick patient has a unique set of symptoms, of apparent genetic origin. Efforts such as The Exome Aggregation Consortium have aggregated exome sequencing data from 60,706 unrelated individuals, excluding individuals diagnosed with severe pediatric diseases {Consortium:2015ib}. This resource can be used to filter out rare mutations arising in healthy individuals, providing a clearer path to the identification of the causal mutation for a disease.

\subsection{Genome wide association studies}

Genome wide association studies (GWAS), linkage studies, and related methods, have produced an ever-growing catalog of many thousands of disease-associated mutations/SNPs. Unlike exome sequencing strategies, which require perfect or almost perfect segregation of a mutation with a disease, these methods rely on a statistical association between commonly occurring mutations and different diseases. As these methodologies allow us to agnostically scan the entire genome for mutations, we see 10 times more genomic mutations associated with diseases (90\%) that lie in the non-coding regions of the genome, and likely result not in gene body, but in gene regulatpry alterations. As eleven thousand non-coding SNPs (versus less than a thousand coding SNPs) have been linked to most human genetic diseases, from developmental to psychiatric, it is critical to address these challenges and develop new approaches to elucidate the impact of non-coding SNPs on pathology and underlying molecular mechanisms.

\section{Organization}

In this work, I explore the intersection of transcriptional regulation and mutations implicated in causing different diseases.

How does a complex enhancer come into being? Moreover, when we see dozens and hundreds of genes turn on in orchestrated fashion to form complicated pathways, and we consider the hundreds of enhancers that orchestrate these shard maneuvers, we can’t but wonder how did all this coordinated regulation logic mutate into existence?

\chapref{chap:mer130} 

contributing to the evolution of the mammalian neocortex, and reveals a gene regulatory code that targets this critical tissue during embryonic development. 

we report the first transposable element family with a 73-fold enrichment in the embryonic neocortex. Outside of embryonic stem cells, no enrichment this large has ever been observed. In addition to being more enriched than any transposable element family in any other set of p300 bound sequences from ChIP-seq experiments in other tissues, including the forebrain at an earlier time-point, MER130 instances are specifically active within the developing neocortex. We identify a preserved code of transcriptional regulatory logic within the MER130 family and demonstrate that it is necessary for MER130’s role as an enhancer.

\chapref{chap:autism} Exome sequencing studies have probed the genetic architecture underlying ASD and have identified hundreds genetic loci harboring variants in ASD probands but not their unaffected family members.

Here we use ChIP-seq, published microarray data, and \emph{in situ} hybridization to show that many high-confidence ASD genes are direct transcriptional targets of Tbr1 in the developing mouse neocortex. This regulation is due to Tbr1 binding many regulatory sequences adjacent to the majority of high-confidence ASD genes, reflecting a dense pattern of regulation. In the future, these Tbr1-bound regulatory sequences may be useful for interpreting non-coding ASD variants. We observe an enrichment for genes differentially expressed in Tbr1 mutants among high-confidence ASD genes, as well as an increase in gene expression in the deep cortical layers of the Tbr1 mutants. This is consistent with the role of Tbr1 as a regulator of these genes and calls attention to deep layer cortical neurons, which have been previously implicated in ASD. Our observations were made during cortical neurogenesis, highlighting the developing neocortex as an area of the brain relevant to ASD. 

Based on these findings, we hypothesized that Tbr1 can be used to select a subset of probable ASD genes that are less tolerant to LoF mutations and could therefore be more relevant to ASD. Testing this hypothesis required a method to compare the relative tolerance of different genes to LoF mutations. To observe the relative selection on each gene, we developed the fraction LoF metric, which measures the fraction of non-reference alleles that are LoF for each gene in the Exome Aggregation Consortium reference population; it controls for length, GC-bias, and other factors influencing the mutation rate. Using this metric, we confirm our hypothesis and show Tbr1 can be used to select a subset of probable ASD genes that are less tolerant to LoF mutations and could be more relevant to ASD. We go on to show that these methods can be used to identify genes which have not been previously implicated in ASD. For example, CTNND2 was recently implicated as a critical gene in autism based on studies of female-enriched multiplex families. Its mouse ortholog has 7 adjacent Tbr1-bound regions, and relatively few LoF mutations in humans. Our methodology highlights a small set of probable ASD genes with similar properties, including well-known cortical genes such as NFIB and ZNF238 (RP58), as well as attractive candidates such as MYT1L and TBL1XR1.


\chapref{chap:zfishSnps}  - Despite the huge financial investment and sheer quantity of GWAS, SNPs are rarely experimentally tested. The functional consequences of these mutations need to be understood 
 - Our first list of 22 SNPs includes associations with many diverse human diseases and traits including myopia, scoliosis, nasopharyngeal carcinoma, attention deficit disorders, and restless legs syndrome.
- All the CNE/SNP pairs associated to these disorders are deeply conserved and can be tested in zebrafish.





