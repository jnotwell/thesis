\chapter{\textit{Cis}-regulation and transcription factors}
\label{chap:cisreg}
%

\textit{Cis}-regulatory elements are clusters of transcription factor binding sites that affect
the rate of transcription of a target gene, often in a tissue-specific manner~\citep{Blanchette2006,Davidson2006,Jeziorska2009}.
While many genes are pleiotropic (having multiple functions), \cis-regulatory elements tend to be more modular~\citep{Carroll2008}.
Thus, \cis-regulatory elements are less constrained and form an ideal substrate for evolution. \textit{Cis}-regulatory
changes have been connected to phenotype evolution in fruit flies and stickleback fish~\citep{Carroll2008, Chan2010}.
\textit{Cis}-regulatory elements have also been connected to human disease~\citep{Gaulton2010, Lettice2003}.  In fact, most of the
genomic regions associated with human diseases and phenotypes by genome-wide association studies (GWAS) do not contain
protein coding genes, and many likely affect gene regulation~\citep{Hindorff2009}.

A typical \cis-regulatory element is on the order of 100 to 1,000 DNA basepairs~\citep{Jeziorska2009}
and contains binding site for multiple transcription factor proteins, thereby mediating combinatorial
control of target gene expression~\citep{Remenyi2004}.
%

\section{Classes of \cis-regulatory elements}
The major classes of \cis-regulatory elements in the genome are promoters, enhancers, silencers, and insulators.
Each class has a different functional role and is correlated with different characteristic epigenetic
modifications~\citep{Maston2006}.
%

Promoters lie proximal to a target gene and interact with distal regulatory elements to control gene expression.
The presence of high levels of tri-methylation of histone 3 at lysine 4 (H3K4me3), acetylation of histone 3
at lysine 27 (H3K27ac), and the absence of mono-methylation of histone 3 at lysine 4 (H3K4me1) are characteristic
of promoters~\citep{Ernst2011}.
%

Enhancers are distal elements that activate gene expression by interacting with gene promoters.  The presence of
high levels of H3K4me1 and H3K27ac are characteristic of active enhancer regions~\citep{Ernst2011}.  Enhancers
are also characterized by sensitivity to DNAse I and are often bound by the transcriptional-coactivator p300~\citep{Kolovos2012}.
Enhancers typically exhibit cell-type specific behavior, and the enhancer repertoires of different cell types show
little overlap~\citep{Ernst2011}.
%

Silencers, or repressors, quench gene activity.  Silencers are less well characterized than enhancers, and few
examples of \cis-acting silencers are known.  It is thought that silencers or transcripts from silencers direct polycomb repressive
complexes 1 and 2 (PRC1 and PRC2) to target regions.  The polycomb complexes drive tri-methylation of H3K27 (H3K27me3), a
repressive mark~\citep{Kolovos2012}.
%

Insulators function as boundaries that block the interaction between an enhancer or silencer and a promoter~\citep{Bushey2008}.
They define separate regions of regulatory control and prevent a \cis-regulatory element that targets one gene from
interfering with another neighboring gene~\citep{Kolovos2012}.  In vertebrates, insulators are commonly bound by the
CCCTC-binding factor (CTCF)~\citep{Kolovos2012}.  Unlike enhancers, regions bound by CTCF are stable across cell types~\citep{Ernst2011}.
%

\section{Identification of \cis-regulatory elements}
%
There are three common methods in use today to identify \cis-regulatory elements in the genome: evolutionary conservation,
enhancer bashing, and chromatin immunoprecipitation followed by high throughput sequencing (ChIP-seq).
%

\subsection{Evolutionary conservation}
Random mutations occur at a rate of approximately \begin{math}10^{-8}\end{math} per basepair~\citep{Xue2009,Roach2010}
when duplicating the genome.  Mutations that occur in functional regions of the genome often interfere with the encoded
function and decrease the fitness of the organism.  The force of natural selection works to remove such deleterious
mutations from the population.  In contrast, mutations that occur in ``junk'' regions of the genome typically have no effect
on organismal fitness and are more likely to be passed to future generations.  Over evolutionary time, this results in a trend
of functional regions of the genome evolving more slowly more than the ``neutral rate'' of junk DNA; thus, functional
regions often exhibit higher sequence identity between related species than do junk regions.  With the availability of
tens of mammalian genomes, this evolutionary principle provides an approach to identify functional elements in the genome
by identifying regions that are more similar across species than is observed for neutral
regions~\citep{Bejerano2004,Siepel2005,Waterston2002}.
%

With the currently available methods and genomes, at least 5\% and up to 10\% of the human genome appears under
evolutionary constraint and hence is putatively functional~\citep{Garber2009,Waterston2002}.  As protein coding genes
comprise only 1.5\% of the genome, most of the functional sequence is non-coding~\citep{Hsu2006}.  The so-called
``conserved non-coding elements'' (CNEs) of the genome
include multiple classes of functional elements, such as non-coding RNAs and origins of replication.  However,
evidence suggests that a great number of the approximately one million (\begin{math}10^{6}\end{math}) CNEs in the human
genome are \cis-regulatory elements~\citep{Bejerano2004,Encode2011}.  Most convincingly, a large functional genomics
screen of conserved elements showed that approximately half of the over 1,000 tested conserved elements act as enhancers
at a single assayed timepoint~\citep{Pennacchio2006}.  Looking at the set of CNEs
as a whole, the elements are enriched near genes that are involved in development and transcription regulation, genes that
require exquisite control of their expression~\citep{Bejerano2004}.
%

\subsection{Enhancer bashing}
The enhancer bashing approach to identify \cis-regulatory elements is to test a large genomic region around a gene
for enhancer activity.  Then, enhancer bashing dissects the large region
into progressively smaller fragments until a sufficiently local region
with regulatory activity is identified~\citep{Luo2008}.  Enhancer bashing is an incremental approach for any gene, but
it is particularly difficult for transcription factors, which tend to reside in ``gene deserts'' and thus have huge
regulatory domains~\citep{McLean2010, Ovcharenko2005}.
%

\subsection{Chromatin immunoprecipitation with high throughput sequencing (ChIP-seq)}
The advent of ultra high throughput sequencing techniques and the availability of assembled reference genomes
empowered a variety of genome-wide assays, with ChIP-seq
being among the most popular and powerful.  ChIP-seq combines chromatin immunoprecipitation with high throughput
sequencing.  It is used to analyze protein interactions with DNA or epigenetic modifications of the DNA~\citep{Park2009}.  First,
chromatin is gathered from a tissue or cell line of interest.  Then, formaldehyde is used to crosslink proteins
to DNA.  The chromatin is sheared into small fragments using sonication.  Next, an antibody to the protein
or histone modification of interest is used to immunoprecipitate the DNA fragments that are bound to the protein
or have the modification.  Crosslinks are removed, and the immunoprecipitated DNA is sequenced.  The sequenced
fragments are mapped to a reference, and ``peaks'' with a surprisingly high number of reads are identified.  The peaks
represent binding sites of the protein to DNA or DNA regions with the histone modification~\citep{Barski2009}.
%

ChIP-seq with antibodies for transcription factors identifies regulatory elements bound by the transcription factor
in the assayed context.  ChIP-seq with antibodies to the enhancer-associated co-activator protein p300 or to the
histone modifications H3K27ac or H3K4me1 have proven to be effective approaches to identify tissue-specific enhancers
in the genome regardless of which transcription factor binds the enhancer~\citep{Cotney2012,Visel2009}.  Importantly, ChIP-seq is context-specific,
meaning that ChIP-seq in liver cells is informative about the liver but not necessarily other contexts.  This stands
in contrast to evolutionary conservation.  Thus, ChIP-seq identifies tissue-specific regulatory elements but is limited by
sample availability.  Evolutionary conservation identifies regulatory elements across a range of tissues, but conservation
itself does not connect a regulatory element to its function.
%

\section{Transcription factors}
%
Approximately 1,500 to 2,000 of the 20,000 protein coding genes in the genome encode transcription factors,
proteins that regulate gene activity by binding to \cis-regulatory elements in a sequence-specific
manner~\citep{Fulton2009,Vaquerizas2009}.  A transcription factor protein contains at least one DNA binding
domain, which has an affinity for DNA.  Transcription factors can be categorized by DNA binding domain, and
proteins with related domains tend to have similar preferences for DNA.  The major families of transcription
factors in the human genome are: zinc finger C2H2 ($\approx$670 members), homeodomain ($\approx$250), basic helix-loop-helix ($\approx$87),
basic leucine zipper ($\approx$51), nuclear hormone receptor ($\approx$50), forkhead ($\approx$50), high mobility group ($\approx$40), and
ETS ($\approx$27)~\citep{Jolma2010}.
%

\subsection{Models of transcription factor binding preferences}
Transcription factors recognize short 6 to 20 basepair motifs in DNA~\citep{Matys2006}.  As many transcription factor binding
motifs show little dependence between positions, they are often modeled using a position frequency matrix (PFM)
that assumes complete position independence~\citep{Stormo2000}.  Each position is modeled as a multinomial probability vector:
\begin{math} \pi_k = < \pi^A_k, \pi^C_k, \pi^G_k, \pi^T_k > \end{math}, where
\begin{math} \pi^T_k = 1 - \pi^A_k - \pi^C_k - \pi^G_k \end{math}, and the frequency matrix strings
the positions together into a matrix:
\[ \begin{array}{r|cccc}
Position & 1 & 2 & \cdots & n \\
\hline
    A & \pi^A_1 & \pi^A_2 & \cdots & \pi^A_n \\
    C & \pi^C_1 & \pi^C_2 & \cdots & \pi^C_n \\
    G & \pi^G_1 & \pi^G_2 & \cdots & \pi^G_n \\
    T & \pi^T_1 & \pi^T_2 & \cdots & \pi^T_n \\
\end{array} \]

To identify matches to a motif in a DNA sequence, the position frequency matrix is transformed into a position
weight matrix (PWM), and the matrix is used to score all substrings in the DNA sequence~\citep{Stormo2000}.  A threshold is defined, and
strings that score above the threshold are considered motif matches and hence putative binding sites.  Two
transformations of a PFM \begin{math}\Pi\end{math} into a PWM \begin{math}\Psi\end{math} are in common use.
The first employs log-likelihood weighting where: $\psi^X_i = log(\pi^X_i)$.  Under this
interpretation, the score of a DNA substring represents its probability of being generated by the PFM model $\Pi$~\citep{Stormo2000}.
The second interpretation employs information content weighting.  The information content of a column is the Kullback-Leibler
divergence from the uniform distribution:
\begin{equation}
IC(<\pi^A,\pi^C,\pi^G,\pi^T>) = 2 + \sum_{j \in \{A,C,G,T\} } \pi^j \cdot log_2(\pi^j)
\end{equation}
With information content weighting, $\psi^X_i = IC(<\pi^A_i,\pi^C_i,\pi^G_i,\pi^T_i>) \times \pi^X_i$.
This scheme explicitly gives more weight to columns with non-uniform base distributions~\citep{Kel2003}.

\subsection{Characterizing transcription factor binding preferences}
A number of experimental techniques allow the characterization of the binding preferences of a transcription factor
and hence the creation of a position weight matrix model.  For many years, the most common method was Systematic Evolution of
Ligands by Exponential Enrichment (SELEX), which begins with a random pool of oligonucleotides and proceeds with a
series of rounds of selection using the transcription factor of interest.  In each round, the oligonucleotides that bind
to the transcription factor are eluted and amplified for the next round of selection~\citep{Djordjevic2007}.  SELEX typically identifies only
the most strongly bound DNA sequences and is unable to characterize the relative preferences of a transcription factor
for all DNA sequences.  More modern techniques like protein binding microarrays enable high throughput and nearly complete
characterization of the \textit{in vitro} binding preferences of a transcription factor.  A protein binding microarray (PBM)
represents all possible DNA 10-mers on a microarray~\citep{Berger2006}.  The transcription factor of interest is tagged and washed over the
microarray.  The relative preference of the factor for all 10-mers is then inferred from the observed binding of the factor to each
of the microarray probes.  Protein binding microarrays have been used to characterize most members of the homeodomain and ETS
transcription factor families from mouse~\citep{Berger2008,Wei2010}.  PBM and other new techniques, such as massively parallel
SELEX~\citep{Jolma2010} and methods based on microfluidics~\citep{Fordyce2010} and microwells~\citep{Hallikas2006}, promise to
characterize the binding preferences of all mouse and human transcription factors in the coming years.
%

Another common approach to infer the binding preferences of a transcription factor is to identify example binding sites
for the factor \textit{in vivo} and to employ motif discovery tools to identify patterns enriched in the example
sequences~\citep{Bailey2006,Tompa2005}.
ChIP-seq provides hundreds to thousands of example binding sites in the genome from which to infer a model.  As ChIP-seq for
more factors becomes available, analysis of the identified binding sites will provide binding models for many factors.
%

\section{Inferring the functions of a set of \cis-regulatory elements}
ChIP-seq and comparative genomics techniques identify sets of \cis-regulatory elements that share a common property, such
as being bound by a particular transcription factor or being derived from an interspersed repeat.  The biological function
of most non-coding regions in the genome are unannotated.  Thus, the biological function of a set of \cis-regulatory elements
is not readily determined by examining the annotation of the bound regions.  We recently developed a technique that leverages
the rich annotation of gene function to infer the biological function of sets of non-coding \cis-regulatory elements~\citep{McLean2010}.
%

The ``Genomic Regions Enrichment of Annotations Tool'' (GREAT) infers the function of a set of genomic regions assuming that:
1) many of the genomic regions share a functional role, and 2) the genomic regions act as \cis-regulatory elements that regulate
nearby genes.  GREAT assigns each gene in the genome a gene regulatory domain that consists approximately of the genomic span
upstream and downstream of the gene transcription start site up to the nearest gene.  Each genomic region in the input set is
assigned to all genes in whose regulatory domain it falls, and the biological annotations of the associated genes are applied
to the genomic region.  GREAT then applies a statistical test to calculate the probability of the observed number of genomic
regions having a given annotation under the null model that the regions were randomly selected from the genome.  The GREAT genomic region
based analysis has been shown to be an effective approach to identifying the biological function of a set of \cis-regulatory
elements~\citep{McLean2010}.
