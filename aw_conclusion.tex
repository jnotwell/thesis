\chapter{Conclusion}
\label{chap:conclusion}
%

\section{Summary}
More of the functional sequence in the human genome is devoted to \cis-regulatory elements,
which control when during development and where in the body genes are active, than
is in protein coding genes~\citep{Encode2011,Hsu2006}.  \textit{Cis}-regulatory elements have been connected to human
disease and evolution~\citep{Hindorff2009,Lettice2003,McLean2011}, but the function of most \cis-regulatory elements in the human genome is
unannotated, and the grammatical rules for how regulatory function is encoded in DNA
sequence are unknown.  This dissertation describes two efforts to improve the annotation of
\cis-regulatory elements in the human genome.

\chapref{chap:prism} describes a breadth-based approach to annotate many regulatory
elements (\figref{fig:prismFig3}A).  It introduces a novel ``excess conservation'' score to identify transcription factor
binding sites conserved across species (\figref{fig:prismFig1}A).  The score explicitly compares the conservation of a
motif match to that of the surrounding basepairs.  Binding site prediction with the excess
conservation score outperforms state of the art methods for 44 of 47 examined transcription
factors (\tabref{tab:prismTableS1}).  The PRISM method combines excess conservation binding site prediction with the
GREAT functional analysis method to annotate \cis-regulatory elements and gene regulatory
networks in the human and mouse genomes (\figref{fig:prismFig3}A).  PRISM was applied with 332 motifs for 300 unique
transcription factors.  Across human and mouse, PRISM annotates 109,748 mostly distal binding
sites in regulatory elements associated with 7,692 different target genes at an overall false
discovery rate of 15\%.  The predictions span a wide range of biological processes (\figref{fig:prismFig3}E).

\chapref{chap:neocortex} describes a deep look at the \cis-regulation of the developing
mouse neocortex at embryonic day E14.5.  A set of 6,629 candidate neocortex enhancers were
identified using ChIP-seq with an antibody to the enhancer-associated co-activator protein
p300 in E14.5 neocortex tissue (\figref{fig:ncxFig1}A).  Eight of ten selected candidates function as neocortex-specific
enhancers when tested in transgenic mouse enhancer assays (\figref{fig:ncxFig2}).  As a set, the candidate enhancers
are enriched near genes expressed in the E14.5 neocortex and involved in neocortex development (\figref{fig:ncxFig1}D).
The candidate enhancers exhibit strong evolutionary conservation (\figref{fig:ncxFig4}A).  In fact, 23\% of the enhancers
are conserved outside of mammals, although the neocortex is a mammalian-specific structure.
This suggests broad co-option of ancient enhancers for the purpose of patterning the neocortex.
In Section \ref{sec:ncxNetworks}, I propose a general method for deriving gene regulatory networks from sets of
tissue-specific enhancers identified by ChIP-seq.  While ChIP-seq in a tissue identifies enhancers associated with
all of the multiple processes occuring simultaneously in the tissue, the combination of binding
site prediction and GREAT functional analysis allows the identification of function-specific
sub-networks (\figref{fig:ncxFig5}B).  Much as PRISM identifies coherent regulatory networks from the ``mass'' of
regulation in the full genome, applying a PRISM-style approach to tissue-specific enhancers
reveals small sub-networks involved in specific processes.  An illustrative example of a
sub-network centered at the transcription factor \textit{Rbpj} demonstrates the concept (\figref{fig:ncxFig5}B).

The importance of \cis-regulation in development, evolution, and disease has become clear
in the genomic era~\citep{Chan2010,Hindorff2009,Lettice2003,McLean2011}.  Comparative genomic analysis has revealed the vast numbers of regulatory
elements in the genome~\citep{Bejerano2004,Siepel2005}, clever application of high throughput sequencing in ChIP-seq
has enabled the identification of all regions in a genome that are bound
by a particular transcription factor or that have a particular epigenetic modification~\citep{Park2009}, and
\textit{in vivo} and \textit{in vitro} enhancer assays have helped characterize \cis-regulatory
function~\citep{Pennacchio2006}.  The computational genomics methods outlined in this dissertation extend such approaches in three important directions.
First, sequence-based computational approaches are not severely limited by practical challenges
like the number of cells in a population of interest or the duration for which a cell population
persists (though partial annotation of such populations does limit GREAT-based analysis).  While a full experimental study of gene regulation in development
would conceptually require separate experiments in many tissues and at many timepoints, computational
approaches enable the study of the full process as it is encoded in the genome.  Second, computational
approaches can extend the simple enumeration of regions from ChIP-seq into regulatory networks
that pinpoint direct regulatory connections between transcription factors and target genes through
\cis-regulatory elements.  Third, computational analysis allows study of \cis-regulatory evolution
that is not readily examined experimentally.  Along with new experimental approaches,
computational analysis promises to continue to improve our understanding of \cis-regulatory elements
and their role in human development, disease, and evolution.

\section{Future work}
\subsection{PRISM}
The PRISM approach relies on the availability of transcription factor motifs, accurate binding site prediction,
and effective functional enrichment analysis of sets of \cis-regulatory regions.  There are opportunities
to extend and improve all three aspects.

The PRISM screen utilizes a large set of 332 quality motifs representing 300 transcription factors, and new methods for charactering
transcription factor binding preferences (e.g. protein binding microarrays) promise to make many additional
motifs available in the near future~\citep{Berger2008}.  Currently available motifs are particularly lacking for the largest
family of transcription factors in the human genome -- the C2H2 zinc fingers (~\figref{fig:prismFig2}A).  As that family has rapidly
expanded in the human lineage and has been understudied~\citep{Emerson2009,Nowick2011}, approaches such as PRISM could reveal many
novel and interesting connections once the motifs are characterized.  In this way, PRISM serves as an
example of how combining a relatively rapid and inexpensive experimental assay (PBM) with a computational
method (PRISM) can guide time-consuming and expensive experimental studies (e.g. gene knockouts, transgenic enhancer assays).

The excess conservation score identifies binding sites that are conserved more than surrounding basepairs.
Thus, it misses true binding sites in both highly conserved regions (which do not stand out against
the surroundings) and non-conserved, lineage-specific binding sites.  While better methods could improve the
sensitivity of conservation-based binding site prediction, the ultimate solution is to enable
effective single-species binding site prediction through a better understanding of \cis-regulatory grammars and
transcription factor combinatorics.  Transcription factor motifs are short and match many more regions in the human genome
than are bound by the transcription factor.  Cross-species comparison is a tool to identify a subset of
motif matches that are likely to be functional binding sites~\citep{Xie2009}.  An understanding of transcription factor interactions,
and thus the longer and more information rich motif of the resulting complex, would reduce the noise in
single-species prediction and eventually eliminate the need to employ cross-species comparisons.  A second option is
to employ other data in addition to sequence.  ChIP-seq, DNAse I open chromatin data, and other data sets highlight
regions of the genome that are likely to function as regulatory elements in a particular tissue.  Such data focus
on the relevant regions of the vast genome and thus remove much of the noise in binding site prediction.  Such experimental
data is most informative for the tissue on which the experiment was performed, but the discovery of marks for poised
regulatory elements suggests that an assay in one tissue is sometimes informative for others~\citep{Cotney2012}.

The GREAT functional enrichment analysis method has proven to be effective for the analysis of a variety of ChIP-seq
data sets~\citep{Lowe2011,RadaIglesias2011,Yoon2011}, and PRISM employs it effectively to annotate sets of binding site predictions.  GREAT relies
on the rich annotation of gene function and lifts annotation from genes to associated genomic regions.  This
approach has been necessary due to the lack of annotation of non-coding loci in the genome.  However,
recent advances in ChIP-seq and other assays have begun to provide direct, rich annotation of non-coding loci.
This enables a different type of statistical test to infer the function of a set of regulatory elements.  Rather
than examining the annotations of the genes nearby the regulatory elements, it now is possible to examine the direct
annotation of the elements~\citep{Chikina2012}.  I apply an initial version of this approach in \chapref{chap:neocortex}
by examining the overlap of the candidate neocortex enhancers with annotated sets of genomic regions including
DNAse I open chromatin regions and other p300 ChIP-seq datasets.  Approaches to identify genome-wide correlations of genomic
regions will supplement approaches such as GREAT that depend on gene annotation.

\subsection{Gene regulatory networks in neocortex and beyond}
Both functional and computational genomics have advantages, and I argue that a synthesis of both is
needed to characterize gene regulatory networks in high-throughput.  In this
dissertation, I propose just such a synthesis.  Functional genomics ChIP-seq assays using antibodies to
general marks of \cis-regulatory elements provide a catalog of active regulatory elements.
Analyzing the catalogs with the computational tools developed for PRISM begins to build regulatory networks
that identify how genes interact to regulate neocortex development.  I illustrate the approach with a regulatory
network centered on the transcription factor \textit{Rbpj}.

To extend the proposed approach beyond a single transcription factor will require the development of appropriate
null models for functional enrichment and network connectivity.  GREAT analysis of binding site predictions in active
regulatory elements must account for the fact that the highlighted regulatory elements have an underlying functional
bias as a result of being identified in tissue-specific ChIP-seq.  As a first approach, the GREAT analysis could be modified to consider a partial genome of the active
regulatory regions.  Rather than having a null hypothesis that the binding sites are randomly distributed in the
genome, the null hypothesis would assume that the binding sites are randomly distributed in the highlighted regulatory
elements.  Alternatively, the enrichments obtained with real transcription factor motifs could be compared to those
for shuffled motifs.  Shuffled motifs would also provide an appropriate null model for network connectivity.  I expect
the biological networks to have distinct structure compared to networks built from shuffled motifs.  Presumably,
the biological networks will be more highly connected.

The similarity of binding preferences for paralogous transcription factors presents an important challenge in motif-based
study of gene regulatory elements.  The sequence of individual binding sites does not differentiate between transcription
factors with identical or nearly identical motifs.  Gene expression data, such as RNA-seq, would help to eliminate
factors that are not expressed in the tissue under analysis.  This becomes even more powerful as the sub-networks
become more precise and highlight specific sub-structures with specific genes expression profiles.  Protein-protein interaction
data also promises to help resolve ambiguity.  Adjacent binding sites in a \cis-regulatory element are more likely to be for
interacting proteins.  Thus, even for two ambiguous binding sites, protein-protein interaction data would trim the options
to a smaller set of interacting transcription factor pairs.  In somes cases, it might not be possible to determine which
member of a transcription factor paralog family is binding to a particular site.  Presenting ambiguous network predictions
to a domain expert will enable the expert to apply his or her experience and intuition to hypothesize which transcription
factor is the likely relevant one.  Lastly, the ambiguity might in fact represent true biology in that the site is
occupied \textit{in vivo} by multiple factors.

The sensitivity and specificity of ChIP-seq approaches to identify \cis-regulatory elements in a tissue are still
unknown.  For p300, current estimates suggest that around 80\% of the highlighted elements are active \cis-regulatory elements
and that perhaps 20\% of all active \cis-regulatory elements are marked~\citep{Visel2009a}.  It is also unknown how the marks change
over development, specifically how long the marks persist after being applied.
Combining multiple assays for separate enhancer marks at separate timepoints will help provide an answer by
examining the overlap of different marks across timepoints.  A high throughput, inexpensive enhancer assay
will also be important to examine sensitivity and specificity.  Ideally, hundreds of putative enhancers would be
examined to determine how activity correlates with epigenetic modifications.

The study of neocortex development, while inherently interesting, is in many ways a case study for combining
functional and computational genomics to study gene regulation.  Once techniques have been refined, similar
approaches could be applied to study other developing tissues.  This dissertation demonstrates how combining
experimental data with novel computational methods empowers high throughput approaches that extend beyond
what can be achieved with experiment of computation alone.
