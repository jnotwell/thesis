\chapter{Conclusion}
\label{chap:conclusion}

\section{Summary}

\chapref{chap:mer130} describes the first transposable element family with a 73-fold enrichment in the embryonic neocortex. Outside of embryonic stem cells, no enrichment this large has ever been observed. I identified a preserved code of transcriptional regulatory logic within the MER130 family and demonstrate that it is necessary for MER130's role as an enhancer. This study also confirms that MER130 instances bound by EP300 \emph{in vivo} function as enhancers in primary cortical neurons. Furthermore, it addresses the conundrum that MER130 instances not detected by EP300 ChIP-seq function as enhancers in our \emph{in vitro} assay by showing that they occupy closed chromatin regions in both mouse and human brains. These findings suggest that the \emph{in vivo} activity of these elements is actively inhibited by chromatin state. Finally, I interrogate the function of the MER130 family and find that individual instances occur next to genes essential for the normal development of the telencephalon, which gives rise to the neocortex.

From an evolutionary perspective, I extend the known origin of MER130 to the tetrapod ancestor, which far predates the neocortex, a surprising fact given MER130's specificity to the developing neocortex. When I queried vertebrate genome sequences, I failed to find evidence for an actively jumping copy in any extant species. I did, however, find one possible homolog in the coelacanth genome. It is intriguing that this MER130 match lies in the intron of a neural gene and exhibits enhancer activity in cortical neurons, as it may represent the ancestral sequence that seeded the MER130 family.

\chapref{chap:autism} describes the use of ChIP-seq, published microarray data, and \emph{in situ} hybridization to show that many high-confidence ASD genes are direct transcriptional targets of TBR1 in the developing mouse neocortex. This regulation is due to TBR1 binding many regulatory sequences adjacent to the majority of high-confidence ASD genes, reflecting a dense pattern of regulation. In the future, these TBR1-bound regulatory sequences may be useful for interpreting non-coding ASD variants. We observed an enrichment for genes differentially expressed in \emph{Tbr1} mutants among high-confidence ASD genes, as well as an increase in gene expression in the deep cortical layers of the \emph{Tbr1} mutants. This is consistent with the role of TBR1 as a regulator of these genes and calls attention to deep layer cortical neurons, which have been previously implicated in ASD. Our observations were made during cortical neurogenesis, highlighting the developing neocortex as an area of the brain relevant to ASD. 

Based on these findings, I hypothesized that TBR1 can be used to select a subset of probable ASD genes that are less tolerant to LoF mutations and could therefore be more relevant to ASD. Testing this hypothesis required a method to compare the relative tolerance of different genes to LoF mutations. To observe the relative selection on each gene, I developed the fraction LoF metric, which measures the fraction of non-reference alleles that are LoF for each gene in the Exome Aggregation Consortium reference population; it controls for length, GC-bias, and other factors influencing the mutation rate. Using this metric, I confirm our hypothesis and show TBR1 can be used to select a subset of probable ASD genes that are less tolerant to LoF mutations and could be more relevant to ASD. We go on to show that these methods can be used to identify genes which have not been previously implicated in ASD. For example, \emph{CTNND2} was recently implicated as a critical gene in autism based on studies of female-enriched multiplex families. Its mouse ortholog has 7 adjacent TBR1-bound regions, and relatively few LoF mutations in humans. Our methodology highlights a small set of probable ASD genes with similar properties, including well-known cortical genes such as \emph{NFIA}, \emph{NFIB}, \emph{ZBTB18}, \emph{CUX2}, and \emph{LRP6}, as well as attractive candidates such as \emph{MYT1L} and \emph{PBX1}.

Over the past decade, GWAS have uncovered thousands of mutations associated with different human diseases and phenotypes. It is extremely rare, however, for the associated mutations to undergo functional experimentation, leaving this treasure trove of insights into the molecular basis for disease buried as merely a set of statistical associations. Teasing apart the functional consequences of these mutations is even more difficult when the SNPs occupy non-genic portions of the genome, in which approximately 90\% of GWAS SNPs reside. In \chapref{chap:zfishSnps}, I designed a novel computational screen to identify human non-coding SNPs present in deeply conserved non-coding elements (CNE) and report a first list of 22 CNE/SNP pairs associated with a large panel of human diseases and phenotypes, ranging from psychiatric disorders to cancer, that can be studied in zebrafish.

This study demonstrates how the combination of computational predictions with a vertebrate model system can be used to study a non-coding SNP to: (1) reveal its functional impact on \emph{cis}-regulation, (2) determine the mutagenic impact of the risk allele, (3) identify the correct \emph{cis}-regulated gene, and (4) discover a novel and unexpected biological process underpinning the human phenotype. These findings also illustrate the necessity of functional investigation. In the original GWAS studies, SNP rs17421627 was hypothesized to regulate \emph{MEF2C} or \emph{TMEM161}, instead of \emph{miR-9-2} as we demonstrate here. miR-9 is a well-known regulator of neurogenesis, but its impact on angiogenesis \emph{in vivo} was unsuspected. We now report that miR-9 down-regulation is associated to retinal vasculature defects in human.

\section{Future work}

While transposable elements were once thought to represent ``junk DNA," the remnants of random integrations of genetic parasites into our genome, an increasing body of work has revealed that these elements can be co-opted into enhancers and can even re-wire entire pathways in different biological contexts. Giving rise to this capability is the fact that these transposable elements harbor sequences capable of binding transcription factors. One of the key challenges of transcriptional regulation is understanding the grammar of regulatory sequences. While it is unclear whether the rewiring of pathways by transposable elements is a general phenomenon or limited to a few instances, the few published examples provide a vector of attack for understanding regulatory codes.

The present model for enhancers consists of a series of transcription factor binding sites that may have flexible or rigid spacing constraints. Transposable element families that have been co-opted into an enhancer network represent the simplest manifestation of this: a series of co-linear binding sites present in tens or hundreds of regulatory sequences spread throughout the genome. Furthermore, to the extent that the co-option event occurred in the common ancestor of several extant species that utilize the insertions, the signs of selection will be visible in the observed sequences. This will enable the use of comparative genomics techniques for detecting binding sites. By having a true positive set of regulatory codes, even if small in number, methods can be developed and tuned that characterize such regulatory elements, providing the first step towards understanding the regulatory grammar of non-coding DNA sequences.

With such a foundation in place, additional avenues may be explored. Models of co-linear binding sites can be extended to allow for the rearrangement of binding sites within an enhancer. This is particularly fascinating in the context of sequence homology. Current tools are completely under-powered to detect such events, even if such rearrangements preserve biological function. Given a model and understanding of these events, tools like nhmmer, which I used for both transposable element detection and deeply conserved GWAS SNP alignment, could be modified so that aligning sequences pass through each part of the profile hidden Markov model corresponding to a transcription factor binding site, but in a path that involves cycles or looping. Developments like these would provide two means for understanding mutations associated with or suspected of causing different diseases. First, a better understanding of regulatory codes will help distinguish pathogenic from benign variants by determining which base pairs are essential for biological function. Second, better tools resulting from a better understanding of the regulatory grammar will allow for even more sequences containing disease-associated mutations to be mapped to model organisms where they can be functionally tested.

Such developments can allow for progress to be made in the understanding of monogenic diseases, but a far greater challenge arises from diseases such as ASD, which can result from mutations in tens or hundreds of genes. Exome sequencing and whole genome sequencing will lead to the identification of rare mutations of possible large effect, but distinguishing which of these rare mutations are causal for disease, as well as their mechanistic contribution to the etiology, will remain a challenge.  Previously, groups have tried intersecting variants implicated in ASD with protein-protein interaction networks in an attempt to gain a network view of disease and understand the relationships between different mutated genes. My study of TBR1 incorporates information regarding transcriptional regulation and genes harboring loss-of-function mutations in ASD probands to better understand risk genes. In the future, carefully overlapping a greater number of genomic measurements will be essential for providing a biological scaffold to better understand these heterogeneous - both genetically and often phenotypically - diseases. 

