\documentclass[]{article}
\usepackage{lmodern}
\usepackage{amssymb,amsmath}
\usepackage{ifxetex,ifluatex}
\usepackage{fixltx2e} % provides \textsubscript
\ifnum 0\ifxetex 1\fi\ifluatex 1\fi=0 % if pdftex
  \usepackage[T1]{fontenc}
  \usepackage[utf8]{inputenc}
\else % if luatex or xelatex
  \ifxetex
    \usepackage{mathspec}
  \else
    \usepackage{fontspec}
  \fi
  \defaultfontfeatures{Ligatures=TeX,Scale=MatchLowercase}
\fi
% use upquote if available, for straight quotes in verbatim environments
\IfFileExists{upquote.sty}{\usepackage{upquote}}{}
% use microtype if available
\IfFileExists{microtype.sty}{%
\usepackage{microtype}
\UseMicrotypeSet[protrusion]{basicmath} % disable protrusion for tt fonts
}{}
\usepackage{hyperref}
\hypersetup{unicode=true,
            pdfborder={0 0 0},
            breaklinks=true}
\urlstyle{same}  % don't use monospace font for urls
\IfFileExists{parskip.sty}{%
\usepackage{parskip}
}{% else
\setlength{\parindent}{0pt}
\setlength{\parskip}{6pt plus 2pt minus 1pt}
}
\setlength{\emergencystretch}{3em}  % prevent overfull lines
\providecommand{\tightlist}{%
  \setlength{\itemsep}{0pt}\setlength{\parskip}{0pt}}
\setcounter{secnumdepth}{0}
% Redefines (sub)paragraphs to behave more like sections
\ifx\paragraph\undefined\else
\let\oldparagraph\paragraph
\renewcommand{\paragraph}[1]{\oldparagraph{#1}\mbox{}}
\fi
\ifx\subparagraph\undefined\else
\let\oldsubparagraph\subparagraph
\renewcommand{\subparagraph}[1]{\oldsubparagraph{#1}\mbox{}}
\fi

\date{}

\begin{document}

\textbf{Genome-wide screen for deeply conserved GWAS-SNPs reveals
regulatory mutations~underlying human phenotypes}

\textbf{AUTHORS}

Romain Madelaine\textsuperscript{1}, James H.
Notwell\textsuperscript{2}, Gemini Skariah\textsuperscript{1}, Caroline
Halluin\textsuperscript{1}, Charles C. Chen\textsuperscript{2}, Mark A.
Krasnow\textsuperscript{3}, Gill Bejerano\textsuperscript{2,4,5},
Philippe Mourrain\textsuperscript{1,6}

\textbf{AFFILIATIONS}

\textsuperscript{1} Stanford Center for Sleep Sciences and Medicine,
Department of Psychiatry and Behavioral Sciences, Stanford, CA 94305,
USA

\textsuperscript{2} Department of Computer Science, Stanford, CA 94305,
USA

\textsuperscript{3} Howard Hughes Medical Institute and Department of
Biochemistry, Stanford, CA 94305, USA

\textsuperscript{4} Department of Developmental Biology, Stanford, CA
94305, USA

\textsuperscript{5} Division of Medical Genetics, Department of
Pediatrics, Stanford, CA 94305, USA

\textsuperscript{6} INSERM 1024, Ecole Normale Supérieure Paris, 75005,
France

* Correspondence:
\href{mailto:mourrain@stanford.edu}{\nolinkurl{mourrain@stanford.edu}}
and bejerano@stanford.edu

\textbf{KEYWORDS}

GWAS, SNPs, conserved non-coding elements, human diseases, zebrafish
model system, \emph{miR-9,} retinal vascular caliber, blood vessels
development, \emph{MEIS2, SOX6, SPRY2}\textbf{\\
}

\textbf{ABSTRACT }

More than 90\% of human disease-associated mutations lie in non-genic
regions of the genome, presenting a challenge for interpretation. We
report a screen to identify GWAS SNPs embedded in deeply conserved
non-coding DNA elements (CNE) preserved syntenically across vertebrates.
We found 22 CNE/SNP pairs covering a wide range of human diseases and
validated a subset for enhancer activity and SNP regulatory function,
including CNE1/SNP rs17421627 associated with retinal vasculature
defects. Zebrafish CNE1:EGFP transgenics revealed human CNE1 enhancer
activity in the retina. CRISPR/Cas-9 deletion of CNE1 in the zebrafish
genome led to defects in blood vessel development in the retina,
suggesting that CNE1 enhancer is functionally associated with retinal
vascular caliber in humans. Introducing the risk mutation in CNE1
abolished EGFP expression, indicating that SNP rs17421627 is likely the
causal mutation responsible for the human phenotype. Validating our
prediction, we identified the target gene \emph{cis}-regulated by
CNE1/rs17421627: the neurogenesis modulator, microRNA-9. Consistent with
the GWAS studies, miR-9 depletion also led to retinal vasculature
defects, demonstrating that \emph{miR-9} is the conserved gene
functionally associated with retinal blood vessel development. This
study validates our \emph{in silico} approach to identify conserved
enhancers regulated by GWAS SNPs and demonstrates how the combination of
human and zebrafish genetics reveals enhancer function, the regulatory
activity of the SNP, the \emph{cis}-regulated gene, and the biological
processes disrupted in the associated disorders.

\textbf{\\
}

\textbf{INTRODUCTION}

The NHGRI GWAS catalog (http://www.genome.gov/gwastudies/) contains
thousands of tag single nucleotide polymorphisms (SNPs) significantly
associated with human diseases and phenotypes collected from hundreds of
independent studies. More than 90\% of SNPs in this repository lie in
non-genic regions of the genome, and most are thought to highlight
disease-associated modifications to gene regulatory elements
(\protect\hyperlink{_ENREF_8}{Hindorff et al. 2009}). In addition to
being mostly non-genic, reported SNPs may either be the causal mutation
or may gain their association from linkage with the causal SNP
(\protect\hyperlink{_ENREF_32}{Tuupanen et al. 2009};
\protect\hyperlink{_ENREF_22}{Nicolae et al. 2010}). Selecting SNPs
under strong sequence conservation across many species has provided a
useful strategy for identifying the causal mutation from multiple
associated loci (\protect\hyperlink{_ENREF_31}{Spieler et al. 2014}).
Here we report an original screen to identify human GWAS SNPs embedded
in non-coding DNA regions conserved to zebrafish, a vertebrate model
where the role of these conserved SNPs can be studied \emph{in vivo}. In
this screen, we require the non-coding sequences to be conserved
syntenically adjacent to the same target genes. Because sequence
conservation is a hallmark of functional \emph{cis}-regulatory elements
(\protect\hyperlink{_ENREF_7}{Hiller et al. 2013}), and because these
sequences are conserved syntenically, we hypothesized that our screen
would identify causal regulatory mutations that could be tested in
zebrafish.

Our genome-wide screen identified 22 such SNPs covering a wide range of
human traits and diseases. We validated the \emph{cis}-regulatory
function of 5 CNEs and successfully demonstrated that the risk allele
for 2 of the SNPs abolished enhancer activity and likely contributes to
the associated disease. The CNE1/SNP rs17421627 pair has been associated
with retinal vascular caliber defects in two related GWAS studies
searching for novel loci involved in human vascular diseases
(\protect\hyperlink{_ENREF_9}{Ikram et al. 2010};
\protect\hyperlink{_ENREF_29}{Sim et al. 2013}). Changes in retinal
vascular caliber have been linked with increased cardiovascular risk and
are predictive of global vascular pathology. For example, wider venular
and narrower arteriolar calibers were associated with an increased risk
of coronary heart disease and cardiovascular mortality
(\protect\hyperlink{_ENREF_34}{Wang et al. 2007};
\protect\hyperlink{_ENREF_20}{McGeechan et al. 2009b}), while wider
retinal venular caliber also predicted stroke
(\protect\hyperlink{_ENREF_19}{McGeechan et al. 2009a}).

We found that human CNE1 acts as an enhancer and drives expression in
the central nervous system including the retina. We showed that SNP
rs17421627 is a regulatory mutation and that CNE1/rs17421627 regulates
\emph{microRNA-9} expression. miR-9 is a well-known regulator of
neurogenesis (\protect\hyperlink{_ENREF_4}{Coolen et al. 2013}).
\emph{In vivo}, miR-9 has been shown to control the proliferation and
differentiation of neural stem cells (NSCs) and its inhibition
transiently increases their proliferation, ultimately resulting in an
increased number of differentiating neurons
(\protect\hyperlink{_ENREF_1}{Bonev et al. 2011};
\protect\hyperlink{_ENREF_28}{Shibata et al. 2011};
\protect\hyperlink{_ENREF_5}{Coolen et al. 2012}). The physiological
role of miR-9 during angiogenesis \emph{in} \emph{vivo} is unknown, but
miR-9 has been associated with cancer cell vascularization \emph{in
vitro} (\protect\hyperlink{_ENREF_36}{Zhang et al. 2012};
\protect\hyperlink{_ENREF_38}{Zhuang et al. 2012}). Here, we show that
the deletion of CNE1 (∆CNE1) in the zebrafish genome leads to retinal
vasculature defects. Furthermore, \emph{miR-9} knock-out embryos display
a retinal vascular phenotype reminiscent of the risk allele carrier of
rs17421627 in human. Together, our results validate our \emph{in silico}
approach and uncover an unexpected function for miR-9 in the control of
retinal vasculature development. We also demonstrated that SNP
rs17421627 is the causal mutation affecting the function of CNE1 and
responsible for the retinal vascular caliber phenotype. This illustrates
the effectiveness of studying non-coding GWAS SNPs \emph{in vivo} in
zebrafish to reveal their biological contributions to the associated
human disease and phenotypes.

\textbf{\\
}

\textbf{RESULTS}

\textbf{Computational identification of non-coding GWAS SNPs conserved
to zebrafish}

We began by extracting 14,068 references SNP cluster ID (rsID) entries
from the NHGRI GWAS catalog. We padded each SNP with 100bp on either
side, creating a local sequence context to be used when querying the
zebrafish genome. We then removed all regions overlapping any exon to
avoid coding and splicing related mutations, leaving 11,822 GWAS
SNP-embedding non-coding elements. For each human element, we
constructed a profile Hidden Markov Model (HMM) based on multi-species
sequence conservation and used nhmmer
(\protect\hyperlink{_ENREF_35}{Wheeler and Eddy 2013}) to sensitively
query the zebrafish genome. To specifically identify gene regulatory
homologies we then applied a gene synteny filter, requiring that the
human GWAS SNP-embedding region and its top zebrafish alignment are
found near orthologous genes in human and zebrafish (Fig. 1A and
Methods). This resulted in a final list of 22 human GWAS SNPs embedded
in non-coding elements conserved in zebrafish next to orthologous genes
(Fig. 1A; Table S1). The list includes SNPs associated with traits and
diseases ranging from drug responses to metabolic disorders. To estimate
the false discovery rate (FDR), we used the same set of models to query
the reversed but not complemented zebrafish genome, and this resulted in
zero false positive matches. For 8 of the 22 CNE/SNP pairs, we
identified a single syntenic gene, suggesting the identity of the gene
regulated by the CNE containing the GWAS SNP. Finally, all the CNE/SNPs
pairs identified in this screen are strongly conserved during 400
million years of evolution suggesting that they are functional
\emph{cis}-regulators (Fig. S1B and Fig. S2 for an example).

\textbf{Functional \emph{in vivo} validation of CNE enhancer activity
and the regulatory function of associated SNPs}

To determine whether the conserved CNE/SNP pairs discovered by our
screen act as \emph{cis}-regulatory elements \emph{in vivo}, we tested
the transcriptional activity of 8 of 22 human CNEs during early
zebrafish development. We fused the human CNE regions to a basal
promoter driving EGFP expression to establish stable zebrafish
transgenic lines (at least 3 independent integrations for each CNE/SNP).
We validated enhancer function for 5 of these conserved non-coding
elements (CNE1/rs17421627; CNE8/rs1568679; CNE10/rs12431307;
CNE15/rs11190870; CNE18/rs16932455). The lack of activity observed for
the three CNEs (CNE9, CNE16 and CNE17) could be attributed to the
developmental stage we assayed or their identity as repressor or
insulator elements. The CNEs with validated enhancer function drove
restricted EGFP expression in a large variety a cell types (neurons,
glial cells, muscles and notochord), demonstrating their specific
activity (Fig. 1B, C, Fig. 3B, C and Fig. S1A). Based on predicted
target genes and EGFP expression, we likely identified the
\emph{cis}-regulated gene for CNE1/\emph{miR-9-2}, CNE8/\emph{meis2},
CNE10/\emph{spry2,} and CNE18/\emph{sox6}. Finally, we tested the
causality of the SNP mutation for CNE1/rs17421627 and CNE8/rs1568679,
and validated their regulatory function by demonstrating that the risk
allele abolished enhancer activity Fig. 1B and Fig. 3B, C). Together,
these results validated our \emph{in silico} predictions and
demonstrated the approach efficiently identifies causal CNE/SNP pairs
associated with human disease, as well as the genes they
\emph{cis}-regulate.

\textbf{CNE1 enhancer deletion leads to retinal vasculature defects
\emph{in vivo}}

To investigate the function of the conserved CNE1 enhancer in retinal
vasculature formation, we deleted CNE1 in the zebrafish genome taking
advantage of the CRISPR/Cas-9 system
(\protect\hyperlink{_ENREF_33}{Varshney et al. 2015}). We deleted 770bp
of CNE1 (Fig. 2A), including the deeply-conserved region of 473bp
containing the SNP (Fig. S1B and Fig. S2). We used the CNE1 deletion
mutant as a model to study the human CNE1 risk allele phenotype. CNE1
mutants display normal eye~morphology (Fig. 2B), but micro-angiography
of ∆CNE1 larvae revealed a disrupted blood vessel network and vascular
defects in the retina (Fig. 2C), similar to the human phenotype
associated with the risk allele. We characterized this phenotype using
the \emph{kdrl} (\emph{vegfr2}) reporter line labelling the vasculature
(\protect\hyperlink{_ENREF_2}{Chi et al. 2008}) and observed a reduction
of \textasciitilde{}40\% in the number of vessels and branching points
in the ∆CNE1 mutant (Fig. 2D, E). These results indicate that CNE1 is
functionally associated with blood vessel development in the retina.
These data also suggest that SNP rs17421627 may be a causal mutation
affecting CNE1 activity and responsible of the retinal vascular caliber
defects in human.

\textbf{CNE1 enhancer regulates \emph{miR-9-2} expression}

CNE1/SNP rs17421627 occupies open chromatin in human fetal brain cells
suggesting that it is active and regulates the syntenically-conserved
neighboring genes in the central nervous system (CNS) (Fig. S1C).
Indeed, the human CNE1 (948bp) used to establish \emph{CNE1:egfp}
transgenic zebrafish drove EGFP expression in the brain and retina (Fig.
3B, C and Fig. S3C, D). The SNP is located \textasciitilde{}200kb away
of the closest neighboring coding genes, \emph{TMEM161B} and
\emph{MEF2C} in the human genome (Fig. 3A). In both GWAS studies
(\protect\hyperlink{_ENREF_9}{Ikram et al. 2010};
\protect\hyperlink{_ENREF_29}{Sim et al. 2013}), the authors
hypothesized that \emph{MEF2C} was the target because of its role in
cardiovascular development (\protect\hyperlink{_ENREF_17}{Lin et al.
1997}) but the \emph{CNE1:egfp} expression profile did not match
\emph{tmem161b} and \emph{mef2cb} expression (Fig. 3A). \emph{In
silico}, we also predicted \emph{microRNA-9-2} as a putative target gene
of CNE1 (Fig. 1A). \emph{miR-9-2} is located \textasciitilde{}100kb away
and is the closest gene adjacent to CNE1 (Fig. 3A and Fig. S1C). In
addition, \emph{miR-9-2} is conserved in the CNE1 synteny block
(\protect\hyperlink{_ENREF_13}{Kikuta et al. 2007}) and is embedded in
its precursor, LINC00461 in human
(\protect\hyperlink{_ENREF_25}{Rodriguez et al. 2004}). In zebrafish,
\emph{miR-9} is expressed in the brain and retina in a pattern
reminiscent of the \emph{CNE1:egfp} line (Fig. 3A, B and Fig. S3) and
its expression is conserved in the inner nuclear layer of both the
zebrafish and mouse retina (Fig. 3A, D). \emph{miR-9-5}, the zebrafish
ortholog of the human \emph{miR-9-2}, is also the closest conserved
adjacent gene to CNE1 in the zebrafish genome and is expressed in the
retina (Fig. 3D). Consistent with the GWAS studies, \emph{miR-9}
expression is also reported in human retina
(\protect\hyperlink{_ENREF_11}{Karali et al. 2016}).

We next tested how CNE1 can act a \emph{cis}-regulator of
\emph{miR-9-5}. In zebrafish, the \emph{miR-9-5} gene is transcribed in
two precursors expressed similarly to the mature miR-9
(\protect\hyperlink{_ENREF_21}{Nepal et al. 2015}), and the deletion of
CNE1 leads to a reduction in the expression of \emph{miR-9-5} precursors
(Fig. 4A, B). This result indicates that both \emph{pri-} and
\emph{pre-miR-9-5} are co-regulated by CNE1. Consistent with the
\emph{egfp} expression pattern driven by CNE1, we did not observe any
evidence of \emph{tmem161b} and \emph{mef2cb} being \emph{cis}-regulated
by CNE1 (Fig. 4C, D). Although we observed a reduction of
\emph{CNE1:egfp} expression at 3 dpf, we did not notice any obvious
reduction in \emph{miR-9-5} precursor expression earlier during
development, suggesting that either CNE1 is not active, or that other
\emph{miR-9-5} enhancers act redundantly at these stages. We also showed
that the downregulation in \emph{miR-9-5} precursors is correlated to a
reduction of the mature miR-9-5 expression in 31\% of the ∆CNE1 mutant
larvae (Fig. 4E). Contrary to the~\emph{miR-9-2}~knock-out
(\protect\hyperlink{_ENREF_28}{Shibata et al. 2011}), targeted deletion
of \emph{LINC00461} does not affect brain development in mouse
(\protect\hyperlink{_ENREF_23}{Oliver et al. 2015}), suggesting that the
functional sequence of this locus is the microRNA. Together, these data
suggest that the functional conserved sequence \emph{cis}-regulated by
CNE1 in human is \emph{miR-9-2}.

\textbf{SNP rs17421627 disrupts \emph{miR-9} expression }

These results indicate that CNE1 is a \emph{cis}-regulatory element
regulating \emph{miR-9,} and we confirmed cellular co-expression of
\emph{miR-9} in CNE1:EGFP+ cells in the retina (Fig. 5A) and brain (Fig.
S3C, D). We then examined whether the human SNP rs17421627 had an effect
on the regulatory ability of CNE1 by replacing the protective T allele
with the risk G allele in the original \emph{CNE1:egfp} transgene. While
only one base pair was changed, EGFP expression was completely abolished
(Fig. 3B, C), demonstrating the regulatory activity of the SNP.
Together, these results strongly suggest that SNP rs17421627 is a causal
mutation affecting \emph{miR-9-2} expression and responsible for the
human vessel caliber defects.

\textbf{miR-9 depletion leads to retinal vasculature formation defects}

Because miR-9 is a well-known regulator of neurogenesis
(\protect\hyperlink{_ENREF_4}{Coolen et al. 2013}), the putative
involvement in vasculature formation came as a surprise, as we did not
detect any expression in endothelial cells (ECs) or blood vessels (Fig.
S4A, B and Laure Bally-Cuif, personal communication). Consistent with
these observations, \emph{LINC00461}, the \emph{miR-9-2} precursor
(\protect\hyperlink{_ENREF_25}{Rodriguez et al. 2004}), is not expressed
in ECs in the human brain
(\href{http://www.brainrnaseq.org}{www.brainrnaseq.org};
(\protect\hyperlink{_ENREF_37}{Zhang et al. 2016})). Interestingly,
previous \emph{in vitro} studies associated \emph{miR-9} expression with
tumor angiogenesis (\protect\hyperlink{_ENREF_36}{Zhang et al. 2012};
\protect\hyperlink{_ENREF_38}{Zhuang et al. 2012}), suggesting that
miR-9 activity could have a role during physiological retinal
angiogenesis.

To investigate how a downregulation of \emph{miR-9} expression is
correlated to a retinal vascular phenotype,~we analyzed the expression
of the mature miR-9-5 in the retina of CNE1 mutant larvae. miR-9-5
expression is not detected in 88\% of the mutant retinas (Fig. 5B),
indicating that a downregulation of \emph{miR-9} expression may be
responsible the retinal blood vessel defects observed in ∆CNE1 mutants.
This result also suggests that there is less redundancy between
\emph{miR-9-5} enhancers for driving expression in the retina compared
to the brain, and that CNE1 is major transcriptional regulator of
\emph{miR-9-5} expression in the retina. To further support the function
of~miR-9 in retinal angiogenesis \emph{in vivo}, we used the previously
described miR-9 morpholino (\protect\hyperlink{_ENREF_16}{Leucht et al.
2008}; \protect\hyperlink{_ENREF_4}{Coolen et al. 2013}). In zebrafish,
miR-9~morphants display normal eye~morphology (Fig. S4C),~but the
hyaloid vasculature in the retina was strongly reduced and the remaining
vessels were thicker (Fig. 5C). We characterized this reduction in blood
vessel branching complexity and observed a decrease of
\textasciitilde{}50\% in the number of retinal vessels and branching
points (Fig. 5D), indicating that the vascular network organization is
affected in the retina. Together these results show that miR-9 activity
is required for the normal development of the retinal vasculature in
vertebrates, and that \emph{miR-9} downregulation is, at least
partially, responsible for the retinal vasculature phenotype caused by
rs17421627 risk allele in human.

\textbf{\\
}

\textbf{DISCUSSION}

This work discovered 22 human phenotype-associated regulatory SNPs that
are embedded within sequences conserved to zebrafish, a vertebrate
system where we model one such SNP's functional impact and the biology
of the associated human phenotype \emph{in vivo}. With the growth of the
NHGRI GWAS Catalog driven by larger cohort studies implicating more SNPs
that reach genome-wide significance
(\protect\hyperlink{_ENREF_26}{Schizophrenia Working Group of the
Psychiatric Genomics 2014}), we expect the number of GWAS SNPs conserved
to zebrafish and the utility of our approach to grow (Fig. S5).
Determining the biological implications of these mutations, however,
remains a challenge. The overwhelming majority of the GWAS SNPs
identified (90\% of \textasciitilde{}14k SNPs) occupy non-genic portions
of the genome that do not result in an obvious disruption, such as
causing a non-synonymous amino acid change within a protein coding gene,
making them challenging to interpret. Furthermore, GWAS studies report
tag SNPs that may be merely co-segregating markers with the casual
mutation (\protect\hyperlink{_ENREF_32}{Tuupanen et al. 2009};
\protect\hyperlink{_ENREF_22}{Nicolae et al. 2010}).

Our screen takes on these challenges. The evolutionary distance between
human and zebrafish can reveal non-coding SNPs of likely importance for
gene \emph{cis}-regulation (\protect\hyperlink{_ENREF_7}{Hiller et al.
2013}). A hallmark of \emph{cis}-regulatory sequence conservation is
that it persists alongside the gene or genes it regulates, even in
species as diverged as human and zebrafish, a property which we
exploited in our computational screen. This allowed us to include very
sensitive alignments with an estimated zero false-positives. By
identifying aligning non-coding regions adjacent to orthologous genes,
we also predict the putative target genes whose expression may be
modulated by the regulatory SNPs. For 8 of the 22 CNE/SNP pairs, we
identified a single gene in conserved synteny, meaning that we have
identified a strong candidate gene related to the GWAS SNP.

While our approach cannot comprehensively address the problem of
interpreting all non-genic GWAS SNPs, it provides a tractable approach
for functionally studying a subset of them in zebrafish. The first 22
GWAS SNPs revealed by our genomic screen are associated with a wide
variety of human phenotypes and diseases, which range from drug
responses to cancer. Studying these polymorphic sites in zebrafish
represents a first step in deciphering the biological mechanisms
underpinning the human traits. In addition to CNE1/rs17421627, we
validated the specific transcriptional activity of 4 other conserved
CNEs and likely identify the \emph{cis}-regulated gene for 3 them. CNE18
drives EGFP expression in retinal ganglion cells of the retina and in
muscles of the trunk, where \emph{sox6} mRNA is also expressed and known
to be important during development
(\protect\hyperlink{_ENREF_10}{Jackson et al. 2015}). CNE10 activity is
observed in the notochord, and \emph{spry2}, the predicted
\emph{cis}-regulated gene, is known to be important for the formation of
this structure (\protect\hyperlink{_ENREF_30}{Sivak et al. 2005}).
Furthermore, \emph{meis2} mRNA co-localizes with EGFP driven by CNE8 in
the zebrafish brain, and the introduction of the risk allele, associated
to antipsychotic treatment response, abolishes its activity. Further
investigations will be necessary to shed the light on the role of these
CNE/SNP pairs in the associated human disorders. This study also
demonstrated the efficiency of a targeted deletion of the conserved CNE1
in the zebrafish genome using the CRISPR/Cas-9 system. This genome
editing technique may also be promising for the CNEs without enhancer
activity (CNE9, CNE16 and CNE17), as it may reveal endogenous repressor
functions. Together, these data demonstrated the efficiency of the
zebrafish as a model system to validate the activity of human CNEs and
the regulatory role of SNPs, as well as study the biology of the
associated human diseases. Our study of the \emph{miR-9} CNE/SNP pair in
zebrafish illustrates the necessity of working with an animal model
system, in addition to cell culture, to reveal the impact on gene
regulation at the whole organism level.

Our experiments revealed numerous insights on the functional impact and
biology of the \emph{miR-9-}regulating CNE/SNP. First, a short human CNE
can drive a specific expression pattern in space and time during
zebrafish development, demonstrating the specificity of the enhancer.
Second, the mutagenic potential of a risk allele can be directly tested
in fish allowing us to discriminate between a co-segregating variant and
the causal mutation. Third, by comparing the enhancer reporter
expression with the mRNA distribution of the neighboring genes and by
deleting the CNE enhancer in the zebrafish genome, we identified the
\emph{cis}-regulated gene. Finally, our findings revealed that the
microvasculature defect studied in the GWAS is the result of
\emph{miR-9-2} expression downregulation in neural cells and uncover a
function for miR-9 during angiogenesis \emph{in vivo}. This study
illustrates how the combination of human and zebrafish genetics can
reveal unexpected biological processes regulated by causal SNP
mutations.

\textbf{\\
}

\textbf{EXPERIMENTAL PROCEDURES}

\textbf{Identification of zebrafish conserved non-coding GWAS SNPs}

The NHGRI GWAS catalog (http://www.genome.gov/gwastudies/) was most
recently downloaded on October 7, 2014. rsIDs were extracted from the
catalog and mapped to genomic loci in the GRCh37 human reference (hg19)
using the dbSNP 137 database (\protect\hyperlink{_ENREF_27}{Sherry et
al. 2001}) table from the UCSC Genome Browser
(\protect\hyperlink{_ENREF_12}{Karolchik et al. 2014}). Only uniquely
mapping rsIDs were retained. Each SNP was padded with 100bp on each
side, creating a local sequence context to be used when querying the
zebrafish genome. Those elements overlapping any Ensembl gene annotation
(Ensembl release 75; (\protect\hyperlink{_ENREF_6}{Flicek et al.
2014})), padded by 50bp, were excluded to avoid all coding and splicing
related events, but non-coding intronic sequences were not removed.

For each element, the corresponding alignment blocks (from a varying
number of species) were extracted from the UCSC 46-way multiple
alignment (\protect\hyperlink{_ENREF_12}{Karolchik et al. 2014}).
Sequences that had been soft-masked in the UCSC alignment were
hard-masked to prevent alignments between repetitive sequences. A
profile HMM was constructed from the alignment corresponding to each
element using the hmmbuild utility from the HMMER package (version
3.1b1; (\protect\hyperlink{_ENREF_35}{Wheeler and Eddy 2013})). This
model was used to query the zebrafish Zv9 assembly (danRer7) using
nhmmer (version 3.1b1) with default parameters. All alignments were
retained. The human and zebrafish matches were then realigned using
LASTZ (version 1.02.00; -\/-seed=match4 -\/-hspthresh=500
-\/-gappedthresh=500; HoxD55 scoring matrix;
\url{http://www.bx.psu.edu/~rsharris/lastz/}). Only pairs scoring above
2000 were retained.

GWAS SNPs embedded specifically in gene regulatory regions conserved to
zebrafish are expected to lie next to the orthologous gene(s) in both
human and zebrafish. This property is known as synteny. Ensembl gene
transcripts and ortholog mappings (Ensembl release 75;
(\protect\hyperlink{_ENREF_6}{Flicek et al. 2014})) were used to
identify syntenic alignments. The top-scoring zebrafish alignment for
each human query was labeled as syntenic if there exists an Ensembl gene
transcript within 500kb of the hg19 SNP, as well as an Ensembl gene
transcript corresponding to a human gene ortholog within 500kb of the
zebrafish alignment.

\textbf{CRISPR/Cas-9 deletion of CNE1 in the zebrafish genome}

To delete CNE1, we took advantage of the previously described genome
editing method (\protect\hyperlink{_ENREF_33}{Varshney et al. 2015}). We
used two gRNA targeting the CNE1 locus (5'-CCCGGCGTCCCCCTTCCT-3' and
5'-AGGAAGGGGGACGCCGGG-3'). Co-injection of these gRNA with the
\emph{Cas9} mRNA resulted in a deletion of 770 bp. F0 carriers for the
deletion in the germline were identified by PCR on the clutches and
outcrossed to obtain F1 heterozygous mutants. F2 mutants homozygous for
the deletion were also identified by PCR.

\textbf{Fish lines and developmental conditions}

Embryos were raised and staged according to standard protocols
(\protect\hyperlink{_ENREF_14}{Kimmel et al. 1995}), in accordance with
Stanford University animal care guidelines. The following transgenic
lines were used to visualize endothelial cells:
\emph{Tg(kdrl:ras-mCherry}) (\protect\hyperlink{_ENREF_2}{Chi et al.
2008}) and \emph{Tg(kdrl:egfp}) (\protect\hyperlink{_ENREF_3}{Choi et
al. 2007})\emph{.} Embryos were fixed overnight at 4°C in 4\%
paraformaldehyde/1xPBS, after which they were dehydrated through an
ethanol series and stored at −20°C until use.

\textbf{Plasmid construction and transgenic line establishment}

For the generation of \emph{Tg(CNEx:egfp)} and \emph{Tg(CNExSNP:egfp),}
transgenic lines were made by PCR amplification of human genomic DNA.
PCR products were directionally cloned into the XhoI and BglII sites of
the pTol2-E1b:EGFP vector. Plasmids were injected into one-cell stage
embryos with the Tol2 mRNA transposase
(\protect\hyperlink{_ENREF_15}{Kwan et al. 2007}). For each stable line,
at least 3 independent integrations were used to validate the specific
transcriptional activity or the absence of expression with the risk
allele. For example, for the \emph{Tg(CNE1:egfp),} 4 independent
integrations with an identical expression pattern of the \emph{egfp}
were identified and for the \emph{Tg(CNE1SNP:egfp)}, 9 independent
integrations with no detectable \emph{egfp} expression were identified
by genotyping.

\textbf{\emph{In situ} hybridization and Immunostaining}

\emph{In situ} hybridizations were performed as previously described
(\protect\hyperlink{_ENREF_24}{Oxtoby and Jowett 1993}). \emph{meis2a}
and \emph{sox6} ORF was cloned in a pCS2+ vector using zebrafish cDNA
and antisense DIG labelled probes were transcribed using the linearized
pCS2+ plasmid containing the ORF. For \emph{miR-9} ISH, the previously
described miRCURY detection probe (LNA) hsa--miR-9 (Exiqon) was used
(\protect\hyperlink{_ENREF_16}{Leucht et al. 2008}). For miR-9-5 ISH, a
miRCURY LNA probe was specifically designed by Exiqon. Probes for
\emph{pre-miR-9-5} and \emph{pri-miR-9-5} were amplified by PCR on
genomic DNA to target the previously described regions in the
\emph{miR-9-5} precursor (\protect\hyperlink{_ENREF_21}{Nepal et al.
2015}). \emph{In situs} were revealed using either BCIP and NBT (Roche)
or Fast Red (Roche) as substrates.

Immunohistochemical stainings were performed as previously described
(\protect\hyperlink{_ENREF_18}{Masai et al. 1997}), using either
anti-GFP (1/1000, Torrey Pines Biolabs), anti-HuC/D (1/500, Molecular
Probes), or anti-DsRed (1/500, Clontech) as primary antibodies and Alexa
488 or Alexa 555-conjugated goat anti-rabbit IgG or goat anti-mouse IgG
(1/1000) as secondary antibodies (Molecular Probes). FITC-dextran 2000
kDa (Sigma) was used to visualize the vasculature after injection in the
cardinal vein.

\textbf{Antisense morpholino injection }

For morpholino knockdowns, we used the previously described miR-9 MO
(TCATACAGCTAGATAACCAAAGA) and the control MO (CACCAAACCATATAGAAGTGATA)
(1mM; (\protect\hyperlink{_ENREF_16}{Leucht et al. 2008};
\protect\hyperlink{_ENREF_5}{Coolen et al. 2012})). Embryos were
injected at the one-cell stage with 1-2 nl of the MO.

\textbf{Image acquisition}

Confocal acquisitions were carried out using a Leica SP5 confocal
microscope (Stanford cell science imaging facility). Images were
manipulated using Photoshop (Adobe) software. For quantification,
confocal stacks were analysed using ImageJ software.

\textbf{FIGURE LEGENDS}

\textbf{Figure 1: Identification of GWAS-SNPs embedded in
deeply-conserved non-coding elements and their syntenic genes}

(\textbf{A}) Methodology used to perform the screen and list of
regulatory GWAS SNPs conserved between human and zebrafish. (\textbf{B})
Confocal section of double \emph{in situ}/immunolabelling showing the
overlap in expression of endogenous \emph{meis2a} and EGFP protein in
the \emph{Tg(CNE8:egfp)} adult brain (arrowhead ,zoom in, left panel).
Confocal projections of EGFP immunolabelling in \emph{Tg(CNE8:egfp)}
containing the protective T allele or \emph{Tg(CNE8SNP:egfp)} containing
the risk C allele in larvae at 72 hpf; EGFP expression in the brain is
abolished in the presence of the SNP rs1568679 risk allele. (\textbf{C})
Confocal section of immunolabelling showing EGFP protein expression and
the similarity with \emph{sox6} mRNA expression in muscles of the trunk
in \emph{Tg(CNE18:egfp)} at 24 hpf. Confocal section of double \emph{in
situ}/immunolabelling showing extensive overlap in the expression of
endogenous \emph{sox6} gene and EGFP in \emph{Tg(CNE18:egfp)} retina at
48 hpf. Dorsal view of the brain with anterior up. Lateral view of the
retina. Lateral view of the trunk. Scale bars: 100 µm.

\textbf{Figure 2: CNE1 regulates retinal vasculature formation \emph{in
vivo}}

(\textbf{A}) Schematic of the zebrafish genomic DNA containing CNE1,
which was deleted using the CRISPR/Cas-9 system with a pair of gRNA
(∆CNE1). The deletion is 770 bp long, including the deeply-conserved
region containing the SNP. (\textbf{B, C}) CNE1 homozygous mutants
display normal brain and eye morphology \textbf{(B)}, but the formation
of the hyaloid vasculature in the retina is affected, as revealed by
microangiography using FITC-dextran injection \textbf{(C)}. The retinal
vasculature shown in \textbf{(C)} are from larvae show in \textbf{(B)}.
\textbf{(C)} Confocal projections of mCherry immunolabelling in
\emph{Tg(kdrl:mCherry)} retina at 72 hpf showing hyaloid vasculature
formation in control and ∆CNE1 mutant larvae. (\textbf{D})
Quantification of the hyaloid vasculature network organization observed
in control and ∆CNE1 mutant larvae at 72 hpf. A minimum of 28 retinas
was analyzed for each context. Dorsal view of the brain with anterior
up. Lateral view of the retina. Scale bars: 10 µm. Error bars represent
s.d. *\emph{P}\textless{}0.05, **\emph{P}\textless{}0.001,
***\emph{P}\textless{}0.0005, determined by \emph{t}-test.

\textbf{Figure 3: SNP/rs17421627 is a regulatory mutation of CNE1
enhancer activity }

(\textbf{A}) Schematic of the human and zebrafish genomic DNA containing
CNE1 where SNP rs17421627 is located. Whole-mount \emph{in situ}
hybridization against \emph{tmem161b, mef2cb,} or \emph{miR-9} in larvae
at 72 hpf. (\textbf{B-C}) Confocal projections of EGFP immunolabelling
in \emph{Tg(CNE1:egfp)} containing the protective T allele or
\emph{Tg(CNE1SNP:egfp)} containing the risk G allele (rs17421627) in
larvae at 72 hpf (B) or adult retina immunolabelled with glutamine
synthetase (GS) (C). In \emph{Tg(CNE1:egfp)}, EGFP expression is
detected in cell bodies in the inner nuclear layer of the retina,
telencephalon, optic tectum, and hindbrain. The \emph{Tg(CNE1:egfp)}
expression pattern is reminiscent of \emph{miR-9} expression. EGFP
expression in the CNS is abolished in the presence of the SNP rs17421627
risk allele. (\textbf{D}) Whole-mount \emph{in situ} hybridization
against miR-9 in zebrafish (2 months old) or mouse (P2 stage; upper
panel) or miR-9-5 in zebrafish larvae (lower panel) showing similar
expression in the inner nuclear layer of the retina. Retina (R),
Ganglion cells layer (GCL), Inner nuclear layer (INL), Telencephalon
(T), Optic Tectum (OT), Hindbrain (H). Dorsal view of the brain with
anterior up. Lateral view of the retina. Scale bars: 100 µm.

\textbf{Figure 4: CNE1 regulates \emph{miR-9-5} expression }

Whole-mount \emph{in situ} hybridization against \emph{pri-miR-9-5}
(\textbf{A})\emph{, pre-miR-9-5} (\textbf{B})\emph{, tmem161b}
(\textbf{C})\emph{, mef2c} (\textbf{D}) or the mature miR-9-5
(\textbf{E}) in control and ∆CNE1 larvae at 72 hpf. The expression of
\emph{miR-9-5}, but not \emph{tmem161b} and \emph{mef2c}, is reduced in
the brain of ∆CNE1 mutants. In 31\% of the mutant brains (n=36),
\emph{miR-9-5} expression is reduced (compared to 0\% in the control
larvae (n=33)). Dorsal view of the brain with anterior up. Scale bars:
100 µm.

\textbf{Figure 5: miR-9 controls retinal vasculature development}

(\textbf{A}) Confocal section of double \emph{in situ}/immunolabelling
showing extensive overlap (arrowheads) in the expression of endogenous
\emph{miR-9} and EGFP protein in \emph{Tg(CNE1:egfp)} retina at 72 hpf.
(\textbf{B}) Whole-mount \emph{in situ} hybridization against
\emph{miR-9-5} in control and ∆CNE1 mutant retina at 72 hpf. In 88\% of
the mutant retinas (n=43), \emph{miR-9-5} expression is reduced
(compared to 6\% in the control larvae (n=34)). (\textbf{C}) Confocal
projections of mCherry immunolabelling in \emph{Tg(kdrl:mCherry)} retina
at 72 hpf showing hyaloid vasculature formation in control MO and miR-9
MO injected larvae. (\textbf{D}) Quantification of the hyaloid
vasculature network organization observed in the control MO or the miR-9
MO at 72 hpf. A minimum of 24 retinas was analyzed for each context.
Dorsal view of the brain with anterior up. Lateral view of the retina.
Scale bars: 100 µm (B) or 10 µm (A, C). Error bars represent s.d.
*\emph{P}\textless{}0.05, **\emph{P}\textless{}0.001,
***\emph{P}\textless{}0.0005, determined by \emph{t}-test.

\textbf{Figure S1: Functional validation of conserved CNEs enhancer
activity }

\textbf{(A)} Confocal section of double immunolabelling showing the
expression of EGFP in the notochord of \emph{Tg(CNE10:egfp)} embryos at
24 hpf (HuC/D in red shows the spinal cord). (\textbf{B}) The sequence
surrounding rs17421627 and the protective allele are conserved during
the evolution in human CNE1 (948 bp; hg19.chr5:87,847,186-87,848,133)
and zebrafish CNE1 (860 bp; danRer7.chr5:49,927,812-49,928,671).
(\textbf{C}) The human and zebrafish gene neighborhoods surrounding
CNE1/rs17421627 contain three orthologous and conserved genes:
\emph{TMEM161B}, \emph{MEF2C} and \emph{MIR9-2}. The CNE1/SNP pair is
marked by H3K4me1 and DNase signals in both human fetal brain and
neuronal progenitors. Scale bars: 100 µm.

\textbf{Figure S2:} Multi-species alignment of the deeply conserved
sequence of CNE1 (473 bp; danRer7. chr5:49,928,049-49,928,521).

\textbf{Figure S3: \emph{miR-9} expression in embryonic and adult CNS.}

\textbf{(A)} Whole-mount in situ hybridization against \emph{miR-9}
showing the time course of \emph{miR-9} expression in the developing
embryo at 24, 36 and 48 hpf. (\textbf{B}) Whole-mount in situ
hybridization against \emph{miR-9} in the adult zebrafish brain.
(\textbf{C, D}) Confocal section of double \emph{in
situ}/immunolabelling showing extensive overlap in the expression of
endogenous \emph{miR-9} and EGFP protein in the hindbrain at 72 hpf (C)
or in the ventricular zone of the adult zebrafish telencephalon (D) in
the \emph{Tg(CNE1:egfp)} line. Dorsal view of the brain with anterior
up. Lateral view of the retina. Scale bars: 100 µm (A, B and D) or 10 µm
(C).

\textbf{Figure S4: \emph{miR-9} expression is not detected in
endothelial cells or blood vessels. }

\textbf{(A, B)} Confocal sections of double \emph{in
situ}/immunolabelling in \emph{Tg(kdrl:egfp)} embryos at 48 hpf showing
the absence of co-localization in the expression of the endogenous
\emph{miR-9} and EGFP protein in the retina \textbf{(A)} and the
hindbrain \textbf{(B)}. Dorsal view of the brain with anterior up.
Lateral view of the retina. \textbf{(C)} Schematic representation of the
miR-9 MO binding to microRNA-9. With the control MO, miR-9 is freely
available, allowing the degradation of the mRNA target and revealing the
\emph{miR-9} expression pattern with the specific miR-9 LNA probe. In
presence of the miR-9 MO, \emph{miR-9} is bound by the MO, inhibiting
the mRNA degradation and the binding of the LNA probe. miR-9 morphant
larvae show no obvious defects in the brain and eye morphogenesis
compared to the control MO. Scale bars: 100 µm.

\textbf{Figure S5:} The cumulative number of GWAS SNPs embedded in
non-coding sequences conserved to zebrafish grows each year, driven by
the growth of the GWAS Catalog.

\textbf{Table S1:} Sequence conservation blocks and gene synteny of GWAS
SNPs conserved between human and zebrafish.\textbf{\\
}

\textbf{AUTHORS CONTRIBUTIONS}

J.H.N., C.C.C and G.B designed the computational study, wrote the
algorithms and analyzed the data. R.M and P.M designed and analyzed
\emph{in vivo} experiments. R.M., G.S. and C.H. performed in vivo
experiments. M.A.K. advised on the study. R.M., J.H.N., G.B. and P.M
wrote the manuscript.

\textbf{ACKNOWLEDGEMENT}

We are grateful to Drs. Laure Bally-Cuif, Thomas S. Becker and Ben A.
Barres for critical reading of the manuscript. We thank Steven Sloan,
Louis Leung, Aaron Wenger, Geetu Tuteja, Harendra Guturu and Karen
Mrukfor insightful discussions and help. We are grateful to Neil C. Chi,
David Traver, Nathan Lawson and William S. Talbot for gift of reagents
or helpful discussions. We would also like to thank the Stanford CSIF
Imaging platform.

\textbf{FUNDING}

This work was supported by grants from the National Institute of Health:
R.M., G.S., C.H. and P.M. (NS062798, DK090065 and MH099647), J.H.N.,
C.C.C and G.B (HG005058). R.M was also supported by EMBO Long Term
Fellowship (ALTF 413-2012), J.H.N. by National Science Foundation
Fellowship (DGE-1147470) and a Bio-X Stanford Interdisciplinary Graduate
Fellowship and G.B by a Packard Fellowship.\textbf{\\
REFERENCES}

\protect\hypertarget{_ENREF_1}{}{}Bonev B, Pisco A, Papalopulu N. 2011.
MicroRNA-9 reveals regional diversity of neural progenitors along the
anterior-posterior axis. \emph{Developmental cell} \textbf{20}: 19-32.

\protect\hypertarget{_ENREF_2}{}{}Chi NC, Shaw RM, De Val S, Kang G, Jan
LY, Black BL, Stainier DY. 2008. Foxn4 directly regulates tbx2b
expression and atrioventricular canal formation. \emph{Genes \&
development} \textbf{22}: 734-739.

\protect\hypertarget{_ENREF_3}{}{}Choi J, Dong L, Ahn J, Dao D,
Hammerschmidt M, Chen JN. 2007. FoxH1 negatively modulates flk1 gene
expression and vascular formation in zebrafish. \emph{Developmental
biology} \textbf{304}: 735-744.

\protect\hypertarget{_ENREF_4}{}{}Coolen M, Katz S, Bally-Cuif L. 2013.
miR-9: a versatile regulator of neurogenesis. \emph{Frontiers in
cellular neuroscience} \textbf{7}: 220.

\protect\hypertarget{_ENREF_5}{}{}Coolen M, Thieffry D, Drivenes O,
Becker TS, Bally-Cuif L. 2012. miR-9 controls the timing of neurogenesis
through the direct inhibition of antagonistic factors.
\emph{Developmental cell} \textbf{22}: 1052-1064.

\protect\hypertarget{_ENREF_6}{}{}Flicek P, Amode MR, Barrell D, Beal K,
Billis K, Brent S, Carvalho-Silva D, Clapham P, Coates G, Fitzgerald S
et al. 2014. Ensembl 2014. \emph{Nucleic acids research} \textbf{42}:
D749-755.

\protect\hypertarget{_ENREF_7}{}{}Hiller M, Agarwal S, Notwell JH,
Parikh R, Guturu H, Wenger AM, Bejerano G. 2013. Computational methods
to detect conserved non-genic elements in phylogenetically isolated
genomes: application to zebrafish. \emph{Nucleic acids research}
\textbf{41}: e151.

\protect\hypertarget{_ENREF_8}{}{}Hindorff LA, Sethupathy P, Junkins HA,
Ramos EM, Mehta JP, Collins FS, Manolio TA. 2009. Potential etiologic
and functional implications of genome-wide association loci for human
diseases and traits. \emph{Proceedings of the National Academy of
Sciences of the United States of America} \textbf{106}: 9362-9367.

\protect\hypertarget{_ENREF_9}{}{}Ikram MK, Sim X, Jensen RA, Cotch MF,
Hewitt AW, Ikram MA, Wang JJ, Klein R, Klein BE, Breteler MM et al.
2010. Four novel Loci (19q13, 6q24, 12q24, and 5q14) influence the
microcirculation in vivo. \emph{PLoS genetics} \textbf{6}: e1001184.

\protect\hypertarget{_ENREF_10}{}{}Jackson HE, Ono Y, Wang X, Elworthy
S, Cunliffe VT, Ingham PW. 2015. The role of Sox6 in zebrafish muscle
fiber type specification. \emph{Skeletal muscle} \textbf{5}: 2.

\protect\hypertarget{_ENREF_11}{}{}Karali M, Persico M, Mutarelli M,
Carissimo A, Pizzo M, Singh Marwah V, Ambrosio C, Pinelli M, Carrella D,
Ferrari S et al. 2016. High-resolution analysis of the human retina
miRNome reveals isomiR variations and novel microRNAs. \emph{Nucleic
acids research} doi:10.1093/nar/gkw039.

\protect\hypertarget{_ENREF_12}{}{}Karolchik D, Barber GP, Casper J,
Clawson H, Cline MS, Diekhans M, Dreszer TR, Fujita PA, Guruvadoo L,
Haeussler M et al. 2014. The UCSC Genome Browser database: 2014 update.
\emph{Nucleic acids research} \textbf{42}: D764-770.

\protect\hypertarget{_ENREF_13}{}{}Kikuta H, Laplante M, Navratilova P,
Komisarczuk AZ, Engstrom PG, Fredman D, Akalin A, Caccamo M, Sealy I,
Howe K et al. 2007. Genomic regulatory blocks encompass multiple
neighboring genes and maintain conserved synteny in vertebrates.
\emph{Genome research} \textbf{17}: 545-555.

\protect\hypertarget{_ENREF_14}{}{}Kimmel CB, Ballard WW, Kimmel SR,
Ullmann B, Schilling TF. 1995. Stages of embryonic development of the
zebrafish. \emph{Dev Dyn} \textbf{203}: 253-310.

\protect\hypertarget{_ENREF_15}{}{}Kwan KM, Fujimoto E, Grabher C,
Mangum BD, Hardy ME, Campbell DS, Parant JM, Yost HJ, Kanki JP, Chien
CB. 2007. The Tol2kit: a multisite gateway-based construction kit for
Tol2 transposon transgenesis constructs. \emph{Dev Dyn} \textbf{236}:
3088-3099.

\protect\hypertarget{_ENREF_16}{}{}Leucht C, Stigloher C, Wizenmann A,
Klafke R, Folchert A, Bally-Cuif L. 2008. MicroRNA-9 directs late
organizer activity of the midbrain-hindbrain boundary. \emph{Nature
neuroscience} \textbf{11}: 641-648.

\protect\hypertarget{_ENREF_17}{}{}Lin Q, Schwarz J, Bucana C, Olson EN.
1997. Control of mouse cardiac morphogenesis and myogenesis by
transcription factor MEF2C. \emph{Science} \textbf{276}: 1404-1407.

\protect\hypertarget{_ENREF_18}{}{}Masai I, Heisenberg CP, Barth KA,
Macdonald R, Adamek S, Wilson SW. 1997. floating head and masterblind
regulate neuronal patterning in the roof of the forebrain. \emph{Neuron}
\textbf{18}: 43-57.

\protect\hypertarget{_ENREF_19}{}{}McGeechan K, Liew G, Macaskill P,
Irwig L, Klein R, Klein BE, Wang JJ, Mitchell P, Vingerling JR, de Jong
PT et al. 2009a. Prediction of incident stroke events based on retinal
vessel caliber: a systematic review and individual-participant
meta-analysis. \emph{American journal of epidemiology} \textbf{170}:
1323-1332.

\protect\hypertarget{_ENREF_20}{}{}McGeechan K, Liew G, Macaskill P,
Irwig L, Klein R, Klein BE, Wang JJ, Mitchell P, Vingerling JR, Dejong
PT et al. 2009b. Meta-analysis: retinal vessel caliber and risk for
coronary heart disease. \emph{Annals of internal medicine} \textbf{151}:
404-413.

\protect\hypertarget{_ENREF_21}{}{}Nepal C, Coolen M, Hadzhiev Y,
Cussigh D, Mydel P, Steen VM, Carninci P, Andersen JB, Bally-Cuif L,
Muller F et al. 2015. Transcriptional, post-transcriptional and
chromatin-associated regulation of pri-miRNAs, pre-miRNAs and moRNAs.
\emph{Nucleic acids research} doi:10.1093/nar/gkv1354.

\protect\hypertarget{_ENREF_22}{}{}Nicolae DL, Gamazon E, Zhang W, Duan
S, Dolan ME, Cox NJ. 2010. Trait-associated SNPs are more likely to be
eQTLs: annotation to enhance discovery from GWAS. \emph{PLoS genetics}
\textbf{6}: e1000888.

\protect\hypertarget{_ENREF_23}{}{}Oliver PL, Chodroff RA, Gosal A,
Edwards B, Cheung AF, Gomez-Rodriguez J, Elliot G, Garrett LJ, Lickiss
T, Szele F et al. 2015. Disruption of Visc-2, a Brain-Expressed
Conserved Long Noncoding RNA, Does Not Elicit an Overt Anatomical or
Behavioral Phenotype. \emph{Cerebral cortex} \textbf{25}: 3572-3585.

\protect\hypertarget{_ENREF_24}{}{}Oxtoby E, Jowett T. 1993. Cloning of
the zebrafish krox-20 gene (krx-20) and its expression during hindbrain
development. \emph{Nucleic acids research} \textbf{21}: 1087-1095.

\protect\hypertarget{_ENREF_25}{}{}Rodriguez A, Griffiths-Jones S,
Ashurst JL, Bradley A. 2004. Identification of mammalian microRNA host
genes and transcription units. \emph{Genome research} \textbf{14}:
1902-1910.

\protect\hypertarget{_ENREF_26}{}{}Schizophrenia Working Group of the
Psychiatric Genomics C. 2014. Biological insights from 108
schizophrenia-associated genetic loci. \emph{Nature} \textbf{511}:
421-427.

\protect\hypertarget{_ENREF_27}{}{}Sherry ST, Ward MH, Kholodov M, Baker
J, Phan L, Smigielski EM, Sirotkin K. 2001. dbSNP: the NCBI database of
genetic variation. \emph{Nucleic acids research} \textbf{29}: 308-311.

\protect\hypertarget{_ENREF_28}{}{}Shibata M, Nakao H, Kiyonari H, Abe
T, Aizawa S. 2011. MicroRNA-9 regulates neurogenesis in mouse
telencephalon by targeting multiple transcription factors. \emph{The
Journal of neuroscience : the official journal of the Society for
Neuroscience} \textbf{31}: 3407-3422.

\protect\hypertarget{_ENREF_29}{}{}Sim X, Jensen RA, Ikram MK, Cotch MF,
Li X, MacGregor S, Xie J, Smith AV, Boerwinkle E, Mitchell P et al.
2013. Genetic loci for retinal arteriolar microcirculation. \emph{PloS
one} \textbf{8}: e65804.

\protect\hypertarget{_ENREF_30}{}{}Sivak JM, Petersen LF, Amaya E. 2005.
FGF signal interpretation is directed by Sprouty and Spred proteins
during mesoderm formation. \emph{Developmental cell} \textbf{8}:
689-701.

\protect\hypertarget{_ENREF_31}{}{}Spieler D, Kaffe M, Knauf F, Bessa J,
Tena JJ, Giesert F, Schormair B, Tilch E, Lee H, Horsch M et al. 2014.
Restless legs syndrome-associated intronic common variant in Meis1
alters enhancer function in the developing telencephalon. \emph{Genome
research} \textbf{24}: 592-603.

\protect\hypertarget{_ENREF_32}{}{}Tuupanen S, Turunen M, Lehtonen R,
Hallikas O, Vanharanta S, Kivioja T, Bjorklund M, Wei G, Yan J,
Niittymaki I et al. 2009. The common colorectal cancer predisposition
SNP rs6983267 at chromosome 8q24 confers potential to enhanced Wnt
signaling. \emph{Nature genetics} \textbf{41}: 885-890.

\protect\hypertarget{_ENREF_33}{}{}Varshney GK, Pei W, LaFave MC, Idol
J, Xu L, Gallardo V, Carrington B, Bishop K, Jones M, Li M et al. 2015.
High-throughput gene targeting and phenotyping in zebrafish using
CRISPR/Cas9. \emph{Genome research} \textbf{25}: 1030-1042.

\protect\hypertarget{_ENREF_34}{}{}Wang JJ, Liew G, Klein R, Rochtchina
E, Knudtson MD, Klein BE, Wong TY, Burlutsky G, Mitchell P. 2007.
Retinal vessel diameter and cardiovascular mortality: pooled data
analysis from two older populations. \emph{European heart journal}
\textbf{28}: 1984-1992.

\protect\hypertarget{_ENREF_35}{}{}Wheeler TJ, Eddy SR. 2013. nhmmer:
DNA homology search with profile HMMs. \emph{Bioinformatics}
\textbf{29}: 2487-2489.

\protect\hypertarget{_ENREF_36}{}{}Zhang H, Qi M, Li S, Qi T, Mei H,
Huang K, Zheng L, Tong Q. 2012. microRNA-9 targets matrix
metalloproteinase 14 to inhibit invasion, metastasis, and angiogenesis
of neuroblastoma cells. \emph{Molecular cancer therapeutics}
\textbf{11}: 1454-1466.

\protect\hypertarget{_ENREF_37}{}{}Zhang Y, Sloan SA, Clarke LE, Caneda
C, Plaza CA, Blumenthal PD, Vogel H, Steinberg GK, Edwards MS, Li G et
al. 2016. Purification and Characterization of Progenitor and Mature
Human Astrocytes Reveals Transcriptional and Functional Differences with
Mouse. \emph{Neuron} \textbf{89}: 37-53.

\protect\hypertarget{_ENREF_38}{}{}Zhuang G, Wu X, Jiang Z, Kasman I,
Yao J, Guan Y, Oeh J, Modrusan Z, Bais C, Sampath D et al. 2012.
Tumour-secreted miR-9 promotes endothelial cell migration and
angiogenesis by activating the JAK-STAT pathway. \emph{The EMBO journal}
\textbf{31}: 3513-3523.

\end{document}
