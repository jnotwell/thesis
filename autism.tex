\documentclass[]{article}
\usepackage{lmodern}
\usepackage{amssymb,amsmath}
\usepackage{ifxetex,ifluatex}
\usepackage{fixltx2e} % provides \textsubscript
\ifnum 0\ifxetex 1\fi\ifluatex 1\fi=0 % if pdftex
  \usepackage[T1]{fontenc}
  \usepackage[utf8]{inputenc}
\else % if luatex or xelatex
  \ifxetex
    \usepackage{mathspec}
  \else
    \usepackage{fontspec}
  \fi
  \defaultfontfeatures{Ligatures=TeX,Scale=MatchLowercase}
\fi
% use upquote if available, for straight quotes in verbatim environments
\IfFileExists{upquote.sty}{\usepackage{upquote}}{}
% use microtype if available
\IfFileExists{microtype.sty}{%
\usepackage{microtype}
\UseMicrotypeSet[protrusion]{basicmath} % disable protrusion for tt fonts
}{}
\usepackage{hyperref}
\hypersetup{unicode=true,
            pdfborder={0 0 0},
            breaklinks=true}
\urlstyle{same}  % don't use monospace font for urls
\usepackage{longtable,booktabs}
\usepackage{graphicx,grffile}
\makeatletter
\def\maxwidth{\ifdim\Gin@nat@width>\linewidth\linewidth\else\Gin@nat@width\fi}
\def\maxheight{\ifdim\Gin@nat@height>\textheight\textheight\else\Gin@nat@height\fi}
\makeatother
% Scale images if necessary, so that they will not overflow the page
% margins by default, and it is still possible to overwrite the defaults
% using explicit options in \includegraphics[width, height, ...]{}
\setkeys{Gin}{width=\maxwidth,height=\maxheight,keepaspectratio}
\IfFileExists{parskip.sty}{%
\usepackage{parskip}
}{% else
\setlength{\parindent}{0pt}
\setlength{\parskip}{6pt plus 2pt minus 1pt}
}
\setlength{\emergencystretch}{3em}  % prevent overfull lines
\providecommand{\tightlist}{%
  \setlength{\itemsep}{0pt}\setlength{\parskip}{0pt}}
\setcounter{secnumdepth}{0}
% Redefines (sub)paragraphs to behave more like sections
\ifx\paragraph\undefined\else
\let\oldparagraph\paragraph
\renewcommand{\paragraph}[1]{\oldparagraph{#1}\mbox{}}
\fi
\ifx\subparagraph\undefined\else
\let\oldsubparagraph\subparagraph
\renewcommand{\subparagraph}[1]{\oldsubparagraph{#1}\mbox{}}
\fi

\date{}

\begin{document}

\textbf{Title Page}

\textbf{Title}

Tbr1 Regulates Autism Risk Genes in the Developing Neocortex

\textbf{Authors}

James H. Notwell\textsuperscript{1}, Whitney E.
Heavner\textsuperscript{2,3}, Siavash Fazel Darbandi\textsuperscript{4},
Sol Katzman\textsuperscript{5}, William L. McKenna\textsuperscript{6},
Christian F. Ortiz-Londono\textsuperscript{6}, David
Tastad\textsuperscript{6}, Matthew J. Eckler\textsuperscript{6}, John
L.R. Rubenstein\textsuperscript{4}, Susan K.
McConnell\textsuperscript{3}, Bin Chen\textsuperscript{6*}, Gill
Bejerano\textsuperscript{1,2,7*}

\textbf{Affiliations}

\begin{enumerate}
\def\labelenumi{\arabic{enumi}.}
\item
  Department of Computer Science, Stanford University.
\item
  Department of Developmental Biology, Stanford University.
\item
  Department of Biology, Stanford University.
\item
  Department of Psychiatry, University of California, San Francisco.
\item
  Center for Bimolecular Science and Engineering, University of
  California, Santa Cruz.
\item
  Department of Molecular, Cell, and Developmental Biology, University
  of California, Santa Cruz.
\item
  Division of Medical Genetics, Department of Pediatrics, Stanford
  University.
\end{enumerate}

* Corresponding authors

\textbf{Contact Information}

Correspondence and requests for materials should be addressed to
\href{mailto:bejerano@stanford.edu}{\nolinkurl{bejerano@stanford.edu}}
and \href{mailto:bchen@ucsc.edu}{\nolinkurl{bchen@ucsc.edu}}.

\textbf{Running Title}

Tbr1 Regulates Autism Risk Genes

\textbf{Keywords}

Autism, gene regulation

\textbf{Abstract}

Exome sequencing studies have identified multiple genes harboring
\emph{de novo} loss-of-function (LoF) variants in individuals with
autism spectrum disorders (ASD), including TBR1, a master regulator of
cortical development. We performed ChIP-seq for Tbr1 during mouse
cortical neurogenesis and show that Tbr1-bound regions are enriched
adjacent to ASD genes. ASD genes were also enriched among genes that are
differentially expressed in \emph{Tbr1} knockouts, which together with
the ChIP-seq data, suggests direct transcriptional regulation. Of the 9
ASD genes examined, 7 were misexpressed in the cortices of \emph{Tbr1}
knockout mice, including 6 with increased expression in the deep
cortical layers. ASD genes with adjacent cortical Tbr1 ChIP-seq peaks
also showed unusually low levels of LoF mutations in a reference human
population and among Icelanders. We then leveraged Tbr1 binding to
identify an appealing subset of candidate ASD genes. Our findings
highlight a TBR1-regulated network of ASD genes in the developing
neocortex that are relatively intolerant to LoF mutations, indicating
that these genes may play critical roles in normal cortical development.

\textbf{\\
}

\textbf{Introduction}

Exome sequencing studies have probed the genetic architecture underlying
ASD by identifying mutations in ASD probands but not their unaffected
family members. From a cohort of 1,043 families assembled from four
previous ASD exome sequencing studies (Iossifov et al. 2012; Kong et al.
2012; Neale et al. 2012; O'Roak et al. 2012) with an additional 56
quartets from the Simons Simplex collection (Willsey et al. 2013),
Willsey and colleagues collected 9 genes with two or more \emph{de novo}
loss-of-function (LoF) mutations in unrelated ASD probands. Because
genes with recurrent \emph{de novo} LoF mutations in ASD probands were
identified as high-confidence ASD genes, we will refer to these 9 genes
as the ``high-confidence (hc) Willsey'' subset. The authors also
identified 122 genes with a single \emph{de novo} LoF mutation among ASD
probands, which we term ``probable (p) Willsey.'' Two subsequent
large-scale studies, which expand the cohorts used in earlier studies,
have implicated additional genes with recurrent \emph{de novo} LoF
mutations among ASD probands. The first study by Iossifov \emph{et al}.
reported 27 genes with recurrent \emph{de novo} likely-gene-disrupting
mutations (we refer to this set as ``hcIossifov'') and an additional 326
genes with a single \emph{de novo} likely-gene-disrupting mutation
(``pIossifov''; Iossifov et al. 2014). In the second large study, De
Rubeis \emph{et al}. reported recurrent \emph{de novo} LoF mutations in
18 genes (``hcDeRubeis'') and a single \emph{de novo} LoF mutation in
257 genes (``pDeRubeis''; De Rubeis et al. 2014; Table 1).

Together, these 3 studies identified a total of 35 high-confidence ASD
genes, providing intriguing glimpses into the genetic and molecular
basis for ASD (Table 1). Functional characterization of these
high-confidence ASD genes has revealed an enrichment for genes that are
expressed embryonically (Iossifov et al. 2014) and are involved in
synapse formation, transcriptional regulation, and chromatin remodeling
(De Rubeis et al. 2014). Furthermore, co-expression networks seeded by
high-confidence ASD genes that are enriched for probable ASD genes
converge on deep-layer projection neurons at midfetal stages of cortical
development (Willsey et al. 2013), while other work has also implicated
the superficial cortical layers (Parikshak et al. 2013). Discovering
additional insights into the developmental and functional mechanisms of
these genes, especially shared roles, remains a significant challenge.
While these three studies also identified hundreds of probable ASD
genes, these represent a combination of true ASD genes and genes with
incidental LoF mutations that do not contribute to ASD, given that
benign LoF variants are observed in healthy individuals (MacArthur and
Tyler-Smith 2010). Many of the non-ASD genes with incidental LoF
mutations may be more tolerant of such mutations. Therefore, another
important challenge is determining which of the probable ASD genes
contribute to ASD.

One of the high-confidence ASD genes identified in all three studies is
\emph{TBR1}, a T-box transcription factor (TF) that plays a critical
role in regulating the differentiation and identity of deep-layer
projection neurons in the developing neocortex (Hevner et al. 2001;
Bedogni et al. 2010; McKenna et al. 2011a; Han et al. 2011b; Leone et
al. 2008). The \emph{de novo} \emph{TBR1} mutations found in ASD
patients can cause changes in the transcriptional regulation and
cellular localization of TBR1, as well as its interactions with
co-regulators such as CASK and FOXP2

(Deriziotis et al. 2014). In humans, patients with microdeletions of the
2q24 region, which encompasses \emph{TBR1}, exhibit intellectual
disability and developmental delay (Traylor et al. 2012). In mice,
\emph{Tbr1} haploinsufficiency results in defective axonal projections
and impairments of social interactions, ultrasonic vocalization,
associative memory, and cognitive flexibility (Huang et al. 2014).
Strikingly, Tbr1 transcriptionally regulates \emph{Grin2b} (Chuang et
al. 2014), another high-confidence ASD gene that encodes a subunit of
the NMDA receptor, a major class of excitatory glutamate receptors in
the central nervous system (Dingledine et al. 1999). These observations
raise the possibility that Tbr1 regulates other ASD genes that are
expressed during cortical development. Here we test this hypothesis by
assessing the binding of Tbr1 near ASD genes using ChIP-seq, examining
their expression in \emph{Tbr1} mutant mice, and analyzing the frequency
of LoF mutations within them in reference human populations of
individuals without ASD.

\textbf{Results}

\textbf{Tbr1 binds near high-confidence ASD genes in the developing
neocortex}

To test the hypothesis that ASD genes are transcriptionally regulated by
Tbr1, we performed ChIP-seq for Tbr1 on mouse whole cortex dissected
from embryonic day 15.5 (E15.5) embryos, a stage at which deep cortical
layers have already been generated and are completing their migration
(McConnell 1991; Molyneaux et al. 2007), and identified 7,324 Tbr1
ChIP-seq peaks (see Methods). These peaks significantly overlap Tbr1
ChIP-seq in N2A cells (Han et al. 2011b; \emph{p}-value: 4.0E-04; see
Methods) and are highly enriched for the known Tbr1 motif (Jolma et al.
2013; \emph{E}-value: 8.3E-101; Supplemental Fig. S1; see Methods). In
addition, our Tbr1 ChIP-seq peaks were enriched for overlapping the
active enhancer marks H3K27ac (\emph{p-}value \textless{} 1.0E-04) and
H3K4me1 (\emph{p-}value \textless{} 1.0E-04), as well as H3K9me3
(\emph{p-}value = 2.0E-03) and H3K27me3 (\emph{p-}value \textless{}
1.0E-04; see Methods), marks associated with repressed chromatin states,
assayed in mouse E14.5 whole brain (The ENCODE Project Consortium 2012);
this suggests possible roles for Tbr1 as both a transcriptional
activator and repressor. When we examined \emph{Grin2b} and
\emph{Auts2}, ASD genes that are regulated by Tbr1 (Bedogni et al. 2010;
Chuang et al. 2014), we observed multiple adjacent Tbr1 ChIP-seq peaks
(Fig. 1A). To associate Tbr1 peaks with their putative target genes, we
used the established Genomic Regions Enrichment of Annotations Tool
(GREAT; McLean et al. 2010) with default parameters (see Methods) and
found the majority of peaks were distal to their associated
transcription start sites (Supplemental Fig. S2). We then mapped each of
the high-confidence and probable ASD genes to their mouse orthologs
(Supplemental Fig. S3; see Methods), and used GREAT to test whether Tbr1
was enriched for binding near the high-confidence ASD genes, given the
number of Tbr1 peaks and the size of the genomic regions used by GREAT
to associate peaks with their adjacent genes. For each set of ASD genes,
we calculated a binomial \emph{p}-value that reflects the significance
of the number of peaks adjacent to these genes, as well as a
hypergeometric \emph{p-}value that reflects the significance of the
number of these genes with adjacent peaks (McLean et al. 2010).

Tbr1 bound 27 regions adjacent to 6 of the 9 hcWillsey genes (67\%,
binomial \emph{p}-value: 1.69E-05, hypergeometric \emph{p}-value:
1.59E-02), 66 regions adjacent to 20 of 27 hcIossifov genes (74\%,
binomial \emph{p}-value: 2.03E-04, hypergeometric \emph{p}-value:
5.90E-07), and 42 regions adjacent to 12 of 17 hcDeRubeis genes (71\%,
binomial \emph{p}-value: 1.35E-03, hypergeometric \emph{p}-value:
2.55E-04; Fig. 1B). In addition, when we merged the high-confidence gene
sets, we found that Tbr1 bound 82 regions adjacent to 25 of the 34
high-confidence ASD genes (74\%, binomial \emph{p}-value: 1.11E-04,
hypergeometric \emph{p}-value: 2.81E-08). We previously performed
ChIP-seq in E14.5 neocortex for p300 (Wenger et al. 2013), a
transcriptional co-activator that marks active enhancers, and ChIP-seq
in E15.5 neocortex for Satb2 (McKenna et al. 2015), another master
regulator of cortical development. The Tbr1 high-confidence ASD gene
enrichment is stronger by either the binomial or hypergeometric test
than the enrichments for p300 or Satb2. Furthermore, when the p300 peaks
overlapped by Tbr1 are removed, the remaining p300 peaks show no
statistical enrichment for 2 of the 3 high-confidence ASD gene lists
(Fig. 1B). In addition, The ENCODE Project Consortium performed ChIP-seq
for 9 transcription factors in different subsets of 20 primary cells and
tissues from mouse (The ENCODE Project Consortium 2012). This includes
brain related tissues, including E14.5 whole brain, 8-week olfactory
bulb, 8-week cerebral cortex, and 8-week cerebellum. Tbr1 is more
enriched for binding adjacent to every high-confidence ASD gene set by
either the binomial or hypergeometric statistic than each of the ENCODE
ChIP-seq experiments (Fig. 1B; Supplemental Table S1). As Tbr1 is
enriched adjacent to each high-confidence gene set individually, as well
as the merged set, we used the merged set of high-confidence ASD genes
for all subsequent analyses. Together, these results suggest that the
enrichment for Tbr1 peaks adjacent to high-confidence ASD genes is not
the consequence of these ASD genes being actively transcribed in the
neocortex, but a specific property of Tbr1.

Giving rise to these enrichments, Tbr1 binding reflects a particularly
dense pattern of regulation. In addition to using GREAT to measure the
enrichment of our Tbr1 ChIP-seq peaks adjacent to a set of genes, we can
also use GREAT to measure the significance of observing a given number
of peaks next to each gene individually. Doing so, we observe as many as
10 Tbr1 ChIP-seq peaks adjacent to a single high-confidence ASD gene
(\emph{Dscam}; GREAT binomial \emph{p­-}value: 3.58E-04), 8 peaks
adjacent to the previously known target \emph{Grin2b} (GREAT binomial
\emph{p-}value: 7.25E-03; Fig. 1A), and an average of more than 2
adjacent Tbr1 ChIP-seq peaks per high-confidence ASD gene overall. In
our previous study of E14.5 mouse neocortex, we observed that
\emph{Auts2} has 29 adjacent p300 ChIP-seq peaks (Wenger et al. 2013).
In this study, we found that \emph{Auts2} has 22 adjacent Tbr1 peaks
(GREAT binomial \emph{p-}value: 1.12E-11; Fig. 1A).

Gene co-expression, protein-protein interactions, and combinations of
these together with candidate genes have been used to construct gene
modules that were enriched for ASD genes. Parikshak \emph{et al}.
constructed gene networks from gene expression data collected by
RNA-seq. Their M3 module was enriched for DNA binding and
transcriptional regulation; it also included TBR1 among 996 genes and
was one of two modules significantly enriched for rare \emph{de novo}
variants from ASD probands and superficial cortical layers (Parikshak et
al. 2013). 662 of our Tbr1 ChIP-seq peaks were enriched for binding
adjacent to genes from this module, but the 228 genes with adjacent
peaks were not significant (GREAT binomial \emph{p}-value: 5.77E-16,
GREAT hypergeometric \emph{p}-value: 0.30). In contrast, Li \emph{et
al.} constructed networks from protein interaction data; they identified
a group of genes, termed module \#13, that included TBR1 among 119
genes, which was one of two modules enriched for ASD genes, and included
genes expressed ubiquitously throughout the brain and in the corpus
callosum (Li et al. 2014). Our Tbr1 ChIP-seq peaks were not enriched for
binding adjacent to genes from this module, but the 65 genes with
adjacent peaks were significant (GREAT binomial \emph{p}-value: 0.24,
GREAT hypergeometric \emph{p}-value: 1.18E-10). Finally, Hormozdiari
\emph{et al.} used a combination of co-expression, protein-protein
interactions, and genes with mutations enriched in ASD and intellectual
disability cases but not controls. Their M1 extended network included
TBR1 among 80 genes and was associated with chromatin remodeling and the
Wnt and Notch signaling pathways (Hormozdiari et al. 2015). Here, we
observed no significant enrichment (GREAT binomial \emph{p}-value: 0.35,
GREAT hypergeometric \emph{p}-value: 0.18), possibly due to the
inclusion of missense mutations or genes mutated in individuals with
intellectual disability. Together, these reflected less consistent
enrichments than those for the high-confidence ASD genes.

\textbf{High-confidence ASD genes are mis-regulated in \emph{Tbr1} KOs }

To determine whether Tbr1 binding affects the expression of putative
target genes, we first analyzed published microarray data to investigate
differences in gene expression between \emph{Tbr1}\textsuperscript{+/+}
and \emph{Tbr1}\textsuperscript{-/-} E14.5 mouse whole neocortex
(Bedogni et al. 2010). Differentially expressed probes corresponded to
1,784 of 21,524 genes (8\%) assayed on the array (see Methods). Tbr1
ChIP-seq peaks were significantly enriched adjacent to the
differentially expressed genes (658 genes, GREAT binomial
\emph{p}-value: 1.59eE-29, GREAT hypergeometric \emph{p-}value:
3.54E-36). After excluding \emph{Tbr1} itself from the merged gene list,
we found that 15 of the 33 high-confidence ASD genes (45\%,
hypergeometric \emph{p}-value: 1.39E-08) showed altered expression in
\emph{Tbr1} mutant cortices at E14.5 (Table 2; see Methods). Overall,
the microarray differentially expressed genes were congruent with the
high-confidence genes that had adjacent Tbr1 ChIP-seq peaks (excluding
\emph{Tbr1}): 12 of 15 differentially expressed high-confidence ASD
genes had at least one adjacent Tbr1 ChIP-seq peak (80\%, hypergeometric
\emph{p}-value: 1.68E-05; Table 2). Together, these results suggest that
Tbr1 binding influences high-confidence ASD gene expression and is
likely a direct transcriptional regulator of high-confidence ASD genes
in the developing neocortex.

Tbr1 regulates the expression of genes in specific layers of the
developing cortex (Bedogni et al. 2010). Because microarray profiling
integrates signals from cells throughout the whole cortex and is not
intended to detect changes in subpopulations of cells, we examined the
spatial patterns of gene expression in \emph{Tbr1} mutant cortices using
\emph{in situ} hybridization. Here, we focused our analyses on 9
high-confidence ASD genes (not including \emph{Tbr1} itself) supported
by at least 3 studies: 7 from the intersection of the 3 high-confidence
ASD gene lists (\emph{ANK2, CHD8, DYRK1A, GRIN2B, KATNAL2, POGZ,
SCN2A}), and an additional 2 omitted by Willsey \emph{et al}.
(\emph{ADNP, ARID1B}) but supported by Iossifov \emph{et al.}, De Rubeis
\emph{et al.}, and earlier studies (O'Roak et al. 2011; 2012; Krumm et
al. 2014).

Tbr1 is expressed soon after cortical neurons begin to differentiate and
is most highly expressed in early-born neurons of the preplate, Cajal
Retzius neurons, and layer 6 (Bulfone et al. 1995; Fig. 2). Our \emph{in
situ} experiments revealed that 4 high-confidence ASD genes that are
expressed throughout the cortical plate are significantly reduced in the
brains of \emph{Tbr1}\textsuperscript{-/-} mice at E15.5: \emph{Arid1b,
Ank2}, \emph{Scn2a1}, and \emph{Grin2b} (Fig. 2A and Supplemental Fig.
S5A). Grain counts were performed to quantify \emph{in situ} differences
in expression (see Methods; Fig. 2A and B). We also examined \emph{Tbr1}
knockout mice at P0, when distinct cortical lamina are visible. We
observed \emph{Arid1b} expression, which is restricted to layer 5, to
significantly decrease in the knockout at P0 (Fig. 2B and Supplemental
Figs. S4 and S5B). In addition, \emph{Ank2}, which is expressed
throughout the cortex, and \emph{Scn2a1}, which is enriched in the deep
cortical layers, both showed an increase in expression throughout the
cortex in the P0 \emph{Tbr1} knockout (Fig. 2B and Supplemental Figs. S4
and S5B). Three additional genes,
\emph{Adnp},~\emph{Dyrk1a}~and~\emph{Pogz}, are expressed diffusely
throughout the cortical layers of controls but specifically increase in
expression in the deep layers of P0 \emph{Tbr1} knockouts (Fig. 2B and
Supplemental Figs. S4 and S5B). Collectively, the \emph{in situ}
hybridization analysis from these 2 time points showed that 7 of the 9
high-confidence ASD genes (excluding \emph{Tbr1}) were misexpressed in
the \emph{Tbr1} knockouts. In addition, the two genes which are not
misexpressed, \emph{Chd8} and \emph{Katnal2}, lacked adjacent Tbr1
ChIP-seq peaks, while all of the misexpressed genes, with the exception
of \emph{Pogz}, had one or more adjacent peaks. Thus, together with the
ChIP-seq data, these results are consistent with the role of Tbr1 as a
direct transcriptional regulator of high-confidence ASD genes in the
developing neocortex.

\textbf{Tbr1 binds near highly co-expressed genes}

In addition to being significantly enriched for binding adjacent to
genes differentially expressed from the Tbr1 knockout microarray
experiments, we also turned to the Allen Brain Atlas to identify
co-expressed genes. We identified human genes with mouse orthologs that
had patterns of expression correlated with TBR1 (r ≥ 0.8). Tbr1 was
highly enriched for binding adjacent to these genes with 157 peaks
adjacent to 51 of 87 genes (59\%, GREAT binomial \emph{p}-value:
5.89E-11, GREAT hypergeometric \emph{p}-value: 9.01E-10), further
supporting the role of Tbr1 as an upstream regulator.

\textbf{Tbr1 target genes are depleted for LoF mutations}

The heterozygous nature of \emph{de novo} LoF mutations (Iossifov et al.
2014), suggests that the affected genes were haploinsufficient and that
putative ASD genes are under strong selective pressure. To test this
idea, we examined the frequency of LoF mutations in ASD genes in a large
control population from the Exome Aggregation Consortium (ExAC), which
aggregated exome sequencing data from 60,706 unrelated individuals,
excluding individuals diagnosed with severe pediatric diseases (Exome
Aggregation Consortium et al. 2015). For each gene in the ExAC Browser,
we computed the proportion of mutations that were LoF (fraction LoF; by
computing the proportion of mutations that are LoF, we control for gene
length and sequence context; see Methods) to assess the frequency of LoF
alleles. We discovered that high-confidence ASD genes were depleted for
LoF mutations (Fig. 3A). Furthermore, \emph{TBR1} was the only
high-confidence ASD gene in which no LoF mutations were observed in the
ExAC population.

As discussed above, the probable ASD genes are thought to be a
combination of true ASD genes, as well as unrelated incidental findings.
We found that the probable ASD genes were skewed toward a higher
fraction of LoF alleles in the ExAC population (Fig. 3A). Previous
estimates suggested that only half of the probable ASD genes represented
true ASD risk genes (Willsey et al. 2013). We used our Tbr1 ChIP-seq
results to divide the probable ASD genes with mouse orthologs into two
groups: those with and without a Tbr1 ChIP-seq peak adjacent to their
ortholog (173 of 464 probable ASD genes had adjacent ChIP-seq peaks;
Supplemental Table S2). We then compared the probable ASD genes having
at least one adjacent Tbr1 ChIP-seq peak to the remaining genes and
found that the genes having adjacent Tbr1 ChIP-seq peak(s) were over
2-fold depleted for LoF mutations (2-sample Wilcoxon \emph{p}-value:
3.87E-06 and fold of medians: 2.28; Fig. 3A). In addition, the genes
having at least one adjacent Tbr1 ChIP-seq peak were nominally depleted
for LoF mutations compared to brain expressed genes (2-sample Wilcoxon
\emph{p}-value: 0.11 and fold of medians: 1.33), although this was a
harsh comparison due to overlap between the two sets. We observed the
same trends when using a different mutation tolerance metric, the
Residual Variation Intolerance Scores (Petrovski et al. 2013; RVIS;
Supplemental Fig. S6; see Methods). Therefore, Tbr1 binding allowed us
to identify an appealing subset of probable ASD genes that were less
tolerant to LoF alleles, as reflected by a lower fraction LoF in the
reference data set (Table 3).

\textbf{Tbr1 target genes have fewer biallelic LoF mutations}

More than a thousand non-essential genes with homozygous or compound
heterozygous LoF mutations have been identified by genomic sequencing of
2,636 individuals and genotyping of an additional 101,584 subjects in
Iceland, who were selected because they were affected by common diseases
of adulthood (Sulem et al. 2015). We investigated whether ASD genes,
particularly those regulated by Tbr1, were less likely to have biallelic
LoF mutations in this population. None of the high-confidence ASD genes
had biallelic LoF mutations in the Icelandic population (Fig. 3B). As
above, we split the lists of probable ASD genes with mouse orthologs
into those with and without a Tbr1 ChIP-seq peak adjacent to their
ortholog (Supplemental Table S2). Genes with adjacent Tbr1 peaks were
less likely to show biallelic LoF mutations when compared to those
without, with odds ratios of 0.37 (Fisher's exact test \emph{p-}value:
1.05E-02; Fig. 3B). These results provide additional evidence that TBR1
regulates a group of ASD genes that are intolerant to LoF mutations.

\textbf{Discussion}

Statistical evidence has suggested that between 300 and 1,000 genes
could confer increased ASD risk (Krumm et al. 2014). Discovering
high-confidence ASD genes is a significant challenge, however, because
recurrent mutations in any given gene are uncommon (Yu et al. 2013).
Recent exome sequencing studies on large cohorts composed of thousands
of individuals have revealed 35 high-confidence ASD genes with recurrent
\emph{de novo} LoF mutations. Among these is TBR1, a master regulator of
cortical development. Our study reveals that many high-confidence ASD
genes have the shared mechanism of being direct transcriptional targets
of Tbr1 in the developing neocortex. This regulation is due to distal
Tbr1 binding and reflects a particularly dense pattern of regulation,
which we have previously found to identify important genes for cortical
development (Wenger et al. 2013). In addition to showing that Tbr1
regulates ASD genes, our study reveals the regulatory sequences that are
bound by Tbr1, which will likely be useful for interpreting non-coding
ASD variants.

Our observations were made during cortical neurogenesis, highlighting
the developing neocortex as a brain region relevant to ASD. The increase
of high-confidence ASD gene expression in the deep layers of the
\emph{Tbr1} mutants is consistent with the role of Tbr1 as a regulator
of deep layer cortical identity and calls attention to deep layer
cortical neurons, which have previously been implicated in ASD (Willsey
et al. 2013). This is in contrast to the less consistent enrichments of
Tbr1 peaks adjacent to genes that were part of modules enriched for ASD
genes from other studies (Parikshak et al. 2013; Li et al. 2014;
Hormozdiari et al. 2015). Tbr1 has known roles in the neocortex as both
an activator (i.e. of \emph{Auts2;} Bedogni et al. 2010) and a repressor
(i.e. of \emph{Fezf2;} Han et al. 2011a; McKenna et al. 2011b), which
explains the up and down-regulation of different genes in \emph{Tbr1}
mutant mice, and is consistent with the enrichments we observed for both
active and repressed chromatin marks. Given that the mutations observed
in ASD probands are presumed LoF, the fact that \emph{Tbr1} acts as a
repressor may be surprising. At E15.5, however, all of the genes with
observed changes in expression were down regulated in the \emph{Tbr1}
mutants, suggesting that early corticogenesis may be a~key time
point~in~ASD pathology. In addition, the aberrant expression of these
genes in the \emph{Tbr1} mutants, up or down, may reflect the importance
of their levels of activity in the deep cortical layers. We also
observed instances of the same gene being activated and repressed by
Tbr1, i.e. \emph{Ank2} and \emph{Scn2a1}, at different time points. In
addition to observing this in a previous study (McKenna et al. 2015),
the fact that we observe the same trend of activation and repression
across RISH \emph{in situ}, DIG \emph{in situ}, and RT-qPCR gives us
confidence that such temporal changes in regulation are possible,
perhaps due to different roles of Tbr1 in different cell types, but
understanding this phenomenon will require further study. Finally, the
changes in expression were observed in homozygous mutant mice, whereas
the LoF mutations observed in ASD probands affect a single allele. Given
the haploinsufficient nature of \emph{TBR1}, we expect the changes to
gene expression in the heterozygous model to be more subtle, yet still
relevant to the pathology of ASD.

TBR1's role as a master regulator of cortical development has long made
it a candidate regulator of ASD genes (Bedogni et al. 2010; Chuang et
al. 2015); here, we measure its binding genome-wide in the developing
cortex and confirm its role in regulating other high-confidence ASD
genes for the first time. The enrichment of Tbr1 ChIP-seq peaks adjacent
to high-confidence ASD genes suggests that Tbr1 can be used to select a
subset of probable ASD genes that are less tolerant to LoF alleles and
could be more relevant to ASD (Table 3). Testing this hypothesis
required a method to compare the relative tolerance of different genes
to mutations. A previous study found that exons that were highly
expressed in the brain and contained relatively few non-synonymous
mutations were enriched for \emph{de novo} mutations found in ASD
probands (Uddin et al. 2014). This methodology, however, failed to
produce significant findings for entire genes. Recurrent \emph{de novo}
LoF mutations found in the same gene among ASD probands are thought to
identify high-confidence ASD genes due to the differing rates of
\emph{de novo} LoF mutations in affected versus unaffected siblings
(Iossifov et al. 2014). In addition, the pattern of LoF mutations
observed in ASD exome sequencing studies suggests that these mutations
are heterozygous (Iossifov et al. 2014), and that the evidence of
selection may be visible by examining allele frequencies in a large
reference population, such as the one compiled by ExAC. We captured
these observations in the fraction LoF metric, which controls for
length, GC-bias, and other factors by measuring the rate of LoF
mutations relative to all mutations. Using this metric, we confirm our
hypothesis and show that Tbr1 can be used to select an appealing subset
of probable ASD genes that are less tolerant to LoF alleles and could be
more relevant to ASD. Against the background of all probable ASD genes,
the mouse orthologs of probable ASD genes with at least one adjacent
Tbr1 ChIP-seq peak are most enriched for encoding proteins located in
the synapse, the same enrichment observed by De Rubeis \emph{et al}.,
across all of the Gene Ontology (GO) (Gene Ontology Consortium 2015);
out of 22 probable ASD gene orthologs annotated with synapse, 18 have at
least one adjacent Tbr1 peak (Bonferroni hypergeometric \emph{p}-value:
1.38E-02; Supplemental Table S3). Above, we examined the enrichment of
Tbr1-ChIP-seq peaks adjacent to the genes in previously discovered
autism gene modules. We can also look at the overlap of only the
probable ASD genes with at least one adjacent Tbr1 ChIP-seq peak and
these different gene modules; we find this subset of genes enriched in
the networks identified by Parikshak \emph{et al.} (15 genes,
hypergeometric \emph{p}-value: 2.66E-04), Li \emph{et al.} (4 genes,
hypergeometric \emph{p}-value: 6.60E-03), and Hormozdiari \emph{et al.}
(8 genes, hypergeometric \emph{p}-value: 5.81E-08), suggesting a level
of molecular convergence. In addition, we observed that high-confidence
ASD genes are strongly depleted for LoF mutations and never have
biallelic LoF mutations in the sequenced Icelandic population (Sulem et
al. 2015). TBR1 is also the only high-confidence ASD gene with no LoF
mutation in the ExAC population, providing additional evidence that this
gene is under strong selective pressure.

Protein-protein interaction networks have previously been proposed to
identify novel ASD candidate genes based on their proximity and
connectivity to high-confidence ASD genes, forming an iterative process
by which genetics and interaction networks mutually inform (Krumm et al.
2014). Based on the current study, being a target of Tbr1 in the
developing cortex and having a low LoF mutation burden can be used as
signals for the prioritization of probable ASD genes for targeted
re-sequencing and study in animal models. Tbr1 binding and the fraction
LoF metric allow for the reinterpretation of even high-confidence ASD
genes. \emph{KATNAL2}, a high-confidence ASD gene for which we find no
evidence of Tbr1 regulation, has a fraction LoF that is almost an order
of magnitude higher than all other high-confidence ASD genes, suggesting
that the recurrent \emph{de novo} mutations in this gene reflect weak
constraint rather than essential function in ASD. A recent study showed
that LoF variants in \emph{KATNAL2} were passed from an unaffected
mother to their unaffected male child, reinforcing this point (Iossifov
et al. 2015). This methodology may even be applied to genes that have
not been previously implicated in ASD. For example, \emph{CTNND2} was
recently implicated as a critical gene in autism based on studies of
female-enriched multiplex families (Turner et al. 2015) but was not
found in any of the gene lists used for the current study. Its mouse
ortholog has 7 adjacent Tbr1-bound regions, and a low LoF burden. Our
methodology highlights a small set of probable ASD genes with similar
properties, including well-known cortical genes such as \emph{NFIA},
\emph{NFIB,} \emph{ZNF238} (\emph{RP58}), \emph{CUX2}, and \emph{LRP6}
as well as attractive candidates such as \emph{MYT1L} and \emph{PBX1}
(Table 3). Together, our findings highlight a TBR1-regulated network of
ASD genes in the developing neocortex that are relatively intolerant to
LoF mutations, indicating that these genes may play critical roles in
normal cortical development.

\textbf{Methods}

\textbf{Animals}

All animal work was carried out in compliance with the University of
California at Santa Cruz IACUC, Stanford University IACUC under approved
protocols \#18487 and \#21758, University of California at San Francisco
IACUC under approved protocol \#AN098262-01I, and institutional and
federal guidelines. The day of vaginal plug detection was designated as
E0.5. The day of birth was designated as P0.

\textbf{ChIP-seq}

Chromatin immunoprecipitation followed by deep sequencing (ChIP-seq) was
performed as previously described (Eckler et al. 2014; McKenna et al.
2011a). Cortices dissected from E15.5 embryos were fixed for 10 min with
1\% formaldehyde and neutralized with glycine. The cells were lysed and
the chromatin was sheared into \textasciitilde{}300 base pair (bp)
fragments. Immunoprecipitation reactions were performed in duplicate
using the Rabbit anti TBR1 (Abcam: ab31940, RRID: AB\_0) antibody, which
was previously validated for ChIP-qPCR (McKenna et al. 2011a).
Sequencing libraries were generated from the ChIP-ed DNA and input DNA
for control using the Illumina TruSeq kit according to the
manufacturer's protocol (Supplemental Table S4). 20 million cells were
used for each ChIP-seq experiment. Sequencing was performed on an
Illumina HiSeq 2000 at the UCSC Genome Technology Center. Sequencing
reads were mapped to the mouse reference genome (NCBI37/mm9) using
version 0.7.12-r1039 of the BWA sampe and aln mapping algorithms (Li and
Durbin 2009) with default parameters. ChIP-seq quality was assessed
using phantompeakqualtools version 2.0
(https://code.google.com/p/phantompeakqualtools/) and all replicates
received the highest quality score (Supplemental Table S4) based on the
relative strand cross-correlation coefficient (RSC). Peaks were called
using MACS version 2.1.0 20140616 with a \emph{p}-value cutoff of 0.01
and merged using the Irreproducible Discovery Rate (IDR) framework (Li
et al. 2011) October 2010 version using a threshold of 0.01 as
previously described
(\url{https://sites.google.com/site/anshulkundaje/projects/idr}).

For computing the overlap with Tbr1 ChIP-seq performed in N2A cells,
reads were last downloaded from the Sequence Read Archive (accessions:
SRR1016884 and SRR1016885) on October 19, 2015. Sequencing reads were
mapped to the mouse reference genome (NCBI37/mm9) using version
0.7.12-r1039 of the BWA samse and aln mapping algorithms (Li and Durbin
2009) with default parameters. The number of reads overlapping our set
of Tbr1 ChIP-seq peaks was compared to the number of reads overlapping
10,000 shuffles of the same peaks across the genome (excluding the UCSC
mm9 gap track).

We downloaded H3K27ac, H3K4me1, H3K27me3, and H3K9me3 histone
modifications assayed in mouse E14.5 whole brain (The ENCODE Project
Consortium 2012). Enrichments with Tbr1 ChIP-seq peaks were computed as
has been done previously (Notwell et al. 2015): we shuffled the Tbr1
ChIP-seq peaks across the genome 10,000 times (excluding the UCSC mm9
gap track). For each shuffle, we counted the number of Tbr1 peaks
overlapped by each histone modification and computed an empirical
\emph{p}-value.

Motif discovery was performed using MEME-ChIP (Machanick and Bailey
2011) version 4.10.2 on the set of IDR peaks called from both
replicates, which were centered over the peak summits from replicate 1
of the MACS ChIP-seq peak calls and trimmed/padded to 201 base pairs,
which was approximately the median peak length.

\textbf{Gene Sets}

Genes with two or more \emph{de novo} LoF mutations in ASD probands
(hcWillsey, hcIossifov, and hcDeRubeis) and genes carrying a single
\emph{de novo} LoF mutation in ASD probands (pWillsey, pIossifov, and
pDeRubeis) were obtained from refs. Willsey et al. 2013; Iossifov et al.
2014; De Rubeis et al. 2014 respectively. Genes were mapped from human
gene symbols to mouse UCSC cluster IDs using mappings from Ensembl
Biomart (Ensembl 78) (Cunningham et al. 2015), the UCSC Genome Browser
(Rosenbloom et al. 2015), and the Mouse Genome Informatics (MGI)
database (Eppig et al. 2015). Ambiguous mappings were excluded, and
mappings were validated using UCSC chains (Rosenbloom et al. 2015). All
9 hcWillsey genes and 117 of 122 pWillsey genes were mapped to their
mouse orthologs; all 27 hcIossifov genes and 312 of 326 pIossifov genes
were mapped to their mouse orthologs; and 17 of 18 hcDeRubeis genes (all
but \emph{SYNGAP1}) and 249 of 257 pDeRubeis genes were mapped to their
mouse orthologs. Additional gene lists were obtained from refs.
Parikshak et al. 2013; Li et al. 2014; Hormozdiari et al. 2015 and
mapped using the same procedures. Finally, genes co-expressed with TBR1
in microarray data from the Allen Human Brain Atlas were downloaded and
mapped to their mouse orthologs. ChIP-seq peaks were associated with
genes using the default basal regulatory domain definition of 5
kilobases (kb) upstream and 1 kb downstream plus extensions in both
directions to the nearest gene's basal domain, up to 1 Megabase (Mb).
Enrichments were computed using GREAT version 2.0.2 (McLean et al.
2010).

\textbf{Transcriptome Profiling}

Transcriptome profiles were last downloaded from GEO (accession:
GSE22371) on March 17, 2015. CEL files were read using the
read.celfiles() function and normalized using the robust multichip
average (RMA) algorithm from version 1.30.0 of the oligo (Carvalho and
Irizarry 2010) R library with default parameters. Differentially
expressed transcripts were identified using limma version 3.22.7
(Ritchie et al. 2015). Probes with a limma reported \emph{p­-}values
less than 0.05 were called differentially expressed, and probes were
mapped to genes using mappings from Ensembl Biomart (Ensembl 78;
Cunningham et al. 2015).

\textbf{Radioactive \emph{in situ} Hybridization}

Subjects in this study were
\emph{Tbr1}\textsuperscript{tm1Jlr}/\emph{Tbr1}\textsuperscript{tm1Jlr}~mice
and their wild-type littermates (RRID: MGI\_3040613). Radioactive
\emph{in situ} hybridization was carried out as previously described
(Frantz et al. 1994), with the following modifications: riboprobes were
transcribed from embryonic mouse neocortex cDNA, and slides were exposed
for a period of three days to three weeks before developing. Primer
pairs were designed to capture as many isoforms as possible
(Supplemental Table S5). Specificity of the probe to the gene of
interest was verified using BLAT (Kent 2002).

\textbf{Grain Counts}

Coronal (P0) and sagittal (E15.5) sections through the frontal cortex
were imaged at 40x magnification such that individual silver grains were
resolved within a single plane. At E15.5, all images were cropped at the
same dimensions to contain only the cortical plate. At P0, images fully
contained either layers 2-5 or layer 6. For all quantifications, images
were chosen with similar cell densities and background levels. Grains
were counted using the ImageJ (Schneider et al. 2012) Analyze Particles
tool. All counts were normalized to background counts in regions of the
brain that were negative for expression of the gene being quantified.
For each gene at each time point and genotype, three to six sections,
encompassing three different samples, were counted.

\textbf{Exome Data}

The latest version 0.3 summary data were last downloaded from the Exome
Aggregation Consortium (ExAC) Browser (Exome Aggregation Consortium et
al. 2015) on October 22, 2015. Variants were obtained from the 60,706
unrelated individuals reported in this database. We used variant
annotations provided in the ExAC download, which were produced by the
Ensembl Variant Effect Predictor v77. For each Ensembl Gene Identifier,
we computed the fraction of non-reference LoF alleles annotated by ExAC,
e.g. nonsense, frameshift, and splice site, normalized to the total
number of non-reference alleles, adjusting for allelic sampling
differences. The fraction LoF score \emph{f} for gene \emph{i} is the
following:

\includegraphics[width=3.08125in,height=1.08125in]{media/image1.emf}

This statistic internally controls for each gene's length, GC bias, and
other confounding factors. The trends and statistics we observed were
robust to the removal of intronic and less harmful variants, as well as
using only the subset of exomes from the National Heart, Lung and Blood
Institute's (NHLBI) Exome Sequencing Project (ESP). RVIS scores computed
from ExAC release 0.3 were last downloaded on October 22, 2015
(http://genic-intolerance.org/data/RVIS\_Unpublished\_ExAC\_May2015.txt),
and genes were mapped from human gene symbols to Ensembl IDs using
mappings from Ensembl Biomart (Ensembl 78; Cunningham et al. 2015).

\textbf{Icelandic KO Data}

The list of genes homozygous or compound heterozygous for LoF mutations
with a minor allele frequency below 2\%, the same criterion used in the
study, were downloaded from Supplementary Table 4 (Sulem et al. 2015)
and mapped to human Ensembl identifiers.

\textbf{Quantitative Real-Time PCR (qRT-PCR)}

qRT-PCR was performed using SYBR Green (Bio-Rad) and a 7900HT Fast
Real-Time PCR System (Applied Biosystems). Gene-specific primers for
high-confidence ASD genes and the \emph{Ef1α} housekeeping gene (HKG)
were designed using the Primer 3 program (Rozen and Skaletsky 2000)
(Supplemental Table S6). The expression levels of the target genes were
normalized relative to the expression levels of \emph{Ef1α} HKG, and
then expression levels between Tbr1-knockouts and wildtype littermates
were compared as previously described (Pfaffl 2001; Darbandi and Franck
2009).

\textbf{Digoxigenin \emph{in situ} Hybridization}

E14.5 mouse embryos were fixed in 4\% paraformaldehyde (PFA) in 1X PBS
overnight at 4°C. The brains were transferred into 30\% sucrose and
incubated overnight at 4°C. Following the sucrose treatment, the brains
were washed in 1X PBS for 5 min at room temperature (RT) and embedded in
Tissue Tek™ O.C.T. compound. P0 mice were perfused with 4\%
paraformaldehyde (PFA) in 1X PBS, followed by an overnight post-fix in
4\% PFA in 1X PBS at 4°C. The brains were transferred into 30\% sucrose
and incubated at 4°C for 48 hours. Following the sucrose treatment, 20
μm samples were sectioned using LEICA SM2000R freezing microtome. 20 μm
sections were obtained from the E14.5 embedded specimen, utilizing a
LEICA CM1900 cryostat, and collected using
Superfrost\textsuperscript{}Plus microscope slides (Fisherbrand).
\emph{In situ} hybridization on frozen tissue sections and digoxigenin
RNA probe labeling were performed according to the procedures previously
described (Long et al. 2003; Wallace and Raff 1999). Hybridized probes
were detected with an AP-conjugated anti-digoxigenin Fab fragment
antibody (1:2000, Roche) and visualized using BM purple (Roche)
substrate system. Antisense riboprobes for high-confidence ASD genes
were prepared as previously described (Cobos et al. 2005; Long et al.
2003) (Supplemental Table S7).

\textbf{Data Access}

Sequencing data have been submitted to the GEO repository under
accession number GSE71384 (for all ChIP-Seq data).

\textbf{Acknowledgements}

We are grateful to Chris Kaznowski for technical help. We would also
like to thank Will Talbot, Natasha O'Brown, and members of the Bejerano
and McConnell labs for helpful comments, especially Aaron Wenger, Geetu
Tuteja, and Dino Leone. This work was supported by a National Science
Foundation Fellowship DGE-1147470 (J.H.N.), a Bio-X Stanford
Interdisciplinary Graduate Fellowship~(J.H.N.), a NIH U01 MH105949 Grant
(G.B.), a NIH R01 MH51864 (S.K.M.), a Stanford Bio-X Interdisciplinary
Initiatives Seed Grant (G.B. and S.K.M.), a NIH R01 MH094589 Grant
(B.C.), Nina Ireland (J.L.R.R.), a NINDS R01 NS34661 (J.L.R.R.), and a
NIMH R37 MH049428 (J.L.R.R.). G.B. is a Packard Fellow and Microsoft
Research Fellow.

\textbf{Disclosure Declaration}

The authors declare no competing financial interests.

\textbf{Author Contributions}

J.H.N, B.C., and G.B. designed the study. J.H.N., W.E.H., S.F.D.,
W.L.M., C.F.O., D.T., M.J.E., and B.C. performed experiments. J.H.N.,
W.E.H., S.F.D., S.K., J.L.R.R., S.K.M., B.C., and G.B. analyzed the
data. J.H.N., W.E.H., S.F.D., J.L.R.R., S.K.M., B.C., and G.B. wrote the
manuscript.\textbf{\\
}

\textbf{Figure Legends}

\textbf{Fig. 1. Tbr1 binds near high-confidence ASD genes} (A)
Regulatory domains of \emph{Grin2b} with 8 adjacent Tbr1 ChIP-seq peaks
and \emph{Auts2} with 22 adjacent Tbr1 ChIP-seq peaks. (B) Significance
of the number of Tbr1 ChIP-seq peaks adjacent to each high-confidence
ASD gene set given the total number of peaks and size of the genomic
regions used to associate peaks with their adjacent genes (the negative
logarithm of the GREAT binomial \emph{p}-value; x-axis) compared to the
significance of the number of high-confidence ASD genes with an adjacent
Tbr1 peak given the total number of genes with an adjacent Tbr1 peak
(negative logarithm of the GREAT hypergeometric \emph{p}-value; y-axis).
Tbr1 enrichments in blue. Enrichment compared to E14.5 neocortex p300
ChIP-seq (red), E15.5 neocortex Satb2 ChIP-seq (orange), and 28 ENCODE
ChIP-seq sets including tissues at different developmental time-points
and primary cell lines (black). Dashed gray lines represent \emph{p} =
0.05 significance level.

\textbf{Fig. 2. Tbr1 is necessary for ASD gene expression in specific
cortical lamina} Radioactive \emph{in situ} hybridization (RISH) of
high-confidence genes at E15.5 (A) and P0 (B) in
\emph{Tbr1}\textsuperscript{+/+} and \emph{Tbr1\textsuperscript{-/-}}
cortices reveal expression differences. RISH expression (lines in shades
of red) corresponds to normalized grain counts (see Methods), and
significance was determined using the 2-sided t-test (see Methods;
\emph{n =} 3 to 6 sections encompassing 3 samples per genotype). E15.5
cortical plate (red), P0 upper layers (brown), and P0 deep layers
(pink). Upper layers correspond to layers 2-5 and deep layers correspond
to layer 6 (see Methods). Error bars represent s.d. *\emph{p}-value
\textless{} 0.05; **\emph{p}-value \textless{} 0.01; ***\emph{p-}value
\textless{} 0.001.

\textbf{Fig. 3.} \textbf{Probable ASD genes that are Tbr1 targets are
more depleted for ExAC LoF mutations and biallelic LoF mutations in
Icelanders} (A) Box-plots depicting the distributions of fraction LoF
scores for each gene from the ExAC reference population (y-axis) for
merged ASD gene lists (x-axis). Probable ASD genes with adjacent Tbr1
ChIP-seq peaks in the developing cortex (blue) have lower fraction LoF
scores than those without an adjacent Tbr1 peak (red). A pseudo-count of
1 LoF allele for 121,412 sampled alleles, the maximum number sampled at
any locus, was included for each gene for visualization purposes.
Significance was determined using the 1-sided 2-sample Wilcoxon Test.
(B) The fraction of genes in each gene list with biallelic mutations in
a study of Icelandic individuals (Sulem et al. 2015). Significance was
determined using the 1-sided Fisher's Exact Test. *\emph{p}-value
\textless{} 0.05; ***\emph{p-}value \textless{} 0.001.

\textbf{\\
}

\textbf{Tables}

\begin{longtable}[]{@{}lllll@{}}
\toprule
~ & Willsey & Iossifov & DeRubeis & Merged\tabularnewline
\midrule
\endhead
High-confidence ASD genes & 9 & 27 & 18 & 35\tabularnewline
Probable ASD genes & 122 & 326 & 257 & 486\tabularnewline
\bottomrule
\end{longtable}

\textbf{Table 1. High-confidence and probable ASD counts} The number of
high-confidence and probable ASD genes identified by previous exome
sequencing studies.

\textbf{\\
}

\begin{longtable}[]{@{}lll@{}}
\toprule
~ & ASD genes & w/ Tbr1 ChIP-seq peak\tabularnewline
\midrule
\endhead
Fraction differentially expressed in \emph{Tbr1\textsuperscript{-/-}} &
15 / 33 (45\%) & 12 / 15 (80\%)\tabularnewline
Hypergeometric \emph{p}-value & 1.39E-08 & 1.68E-05\tabularnewline
\bottomrule
\end{longtable}

\textbf{Table 2. High-confidence ASD genes are mis-regulated in
\emph{Tbr1} knockouts} High-confidence ASD genes are highly enriched
among genes differentially expressed in \emph{Tbr1} KOs in E14.5
neocortex from Bedogni \emph{et al}.

\begin{longtable}[]{@{}llll@{}}
\toprule
~ & LoF \textless{} 5E-05 & LoF \textless{} 2E-04 & LoF \textless{}
1E-03\tabularnewline
\midrule
\endhead
Peaks \emph{p}-value \textless{} 1E-03 & NFIA, NFIB, ZNF238 & CUX2, LRP6
& ~\tabularnewline
Peaks \emph{p}-value \textless{} 1E-02 & MYT1L, PBX1 & FAM8A1, IGSF3,
ZFHX3 & INSC\tabularnewline
Peaks \emph{p}-value \textless{} 5E-02 & PPP1R15B, RELN & CMPK2, NIN,
WNT7B & BRCA1, CECR2, GSDMC\tabularnewline
\bottomrule
\end{longtable}

\textbf{Table 3. Probable ASD genes that are neocortical transcriptional
targets of Tbr1 and less tolerant to LoF alleles} 19 probable ASD genes
enriched for Tbr1 ChIP-seq peaks adjacent to their mouse ortholog based
on the GREAT single-gene binomial test and fraction LoF scores less than
1.0E-3.

\textbf{\\
}

\textbf{References}

Bedogni F, Hodge RD, Elsen GE, Nelson BR, Daza RAM, Beyer RP, Bammler
TK, Rubenstein JLR, Hevner RF. 2010. Tbr1 regulates regional and laminar
identity of postmitotic neurons in developing neocortex. \emph{Proc Natl
Acad Sci USA} \textbf{107}: 13129--13134.

Bulfone A, Smiga SM, Shimamura K, Peterson A, Puelles L, Rubenstein JLR.
1995. T-Brain-1: A homolog of Brachyury whose expression defines
molecularly distinct domains within the cerebral cortex. \emph{Neuron}
\textbf{15}: 63--78.

Carvalho BS, Irizarry RA. 2010. A framework for oligonucleotide
microarray preprocessing. \emph{Bioinformatics} \textbf{26}: 2363--2367.

Chuang H-C, Huang T-N, Hsueh Y-P. 2014. Neuronal excitation upregulates
Tbr1, a high-confidence risk gene of autism, mediating Grin2b expression
in the adult brain. \emph{Front Cell Neurosci} \textbf{8}: 280.

Chuang H-C, Huang T-N, Hsueh Y-P. 2015. T-Brain-1-\/-A Potential Master
Regulator in Autism Spectrum Disorders. \emph{Autism Res} \textbf{8}:
412--426.

Cobos I, Calcagnotto ME, Vilaythong AJ, Thwin MT, Noebels JL, Baraban
SC, Rubenstein JLR. 2005. Mice lacking Dlx1 show subtype-specific loss
of interneurons, reduced inhibition and epilepsy. \emph{Nat Neurosci}
\textbf{8}: 1059--1068.

Cunningham F, Amode MR, Barrell D, Beal K, Billis K, Brent S,
Carvalho-Silva D, Clapham P, Coates G, Fitzgerald S, et al. 2015.
Ensembl 2015. \emph{Nucleic Acids Res} \textbf{43}: D662--9.

Darbandi S, Franck JPC. 2009. A comparative study of ryanodine receptor
(RyR) gene expression levels in a basal ray-finned fish, bichir
(Polypterus ornatipinnis) and the derived euteleost zebrafish (Danio
rerio). \emph{Comp Biochem Physiol B, Biochem Mol Biol} \textbf{154}:
443--448.

De Rubeis S, He X, Goldberg AP, Poultney CS, Samocha K, Cicek AE, Kou Y,
Liu L, Fromer M, Walker S, et al. 2014. Synaptic, transcriptional and
chromatin genes disrupted in autism. \emph{Nature} \textbf{515}:
209--215.

Deriziotis P, O'Roak BJ, Graham SA, Estruch SB, Dimitropoulou D, Bernier
RA, Gerdts J, Shendure J, Eichler EE, Fisher SE. 2014. De novo TBR1
mutations in sporadic autism disrupt protein functions. \emph{Nat
Commun} \textbf{5}: 4954.

Dingledine R, Borges K, Bowie D, Traynelis SF. 1999. The glutamate
receptor ion channels. \emph{Pharmacol Rev} \textbf{51}: 7--61.

Eckler MJ, Larkin KA, McKenna WL, Katzman S, Guo C, Roque R, Visel A,
Rubenstein JLR, Bin Chen. 2014. Multiple conserved regulatory domains
promote Fezf2 expression in the developing cerebral cortex. \emph{Neural
Dev} \textbf{9}: 6.

Eppig JT, Blake JA, Bult CJ, Kadin JA, Richardson JE, Mouse Genome
Database Group. 2015. The Mouse Genome Database (MGD): facilitating
mouse as a model for human biology and disease. \emph{Nucleic Acids Res}
\textbf{43}: D726--36.

Exome Aggregation Consortium, Lek M, Karczewski K, Minikel E, Samocha K,
Banks E, Fennell T, O'Donnell-Luria A, Ware J, Hill A, et al. 2015.
Analysis of protein-coding genetic variation in 60,706 humans.
\emph{bioRxiv} 030338.

Frantz GD, Weimann JM, Levin ME, McConnell SK. 1994. Otx1 and Otx2
define layers and regions in developing cerebral cortex and cerebellum.
\emph{J Neurosci} \textbf{14}: 5725--5740.

Gene Ontology Consortium. 2015. Gene Ontology Consortium: going forward.
\emph{Nucleic Acids Res} \textbf{43}: D1049--56.

Han SSW, Williams LA, Eggan KC. 2011a. Constructing and deconstructing
stem cell models of neurological disease. \textbf{70}: 626--644.

Han W, Kwan KY, Shim S, Lam MMS, Shin Y, Xu X, Zhu Y, Li M, Šestan N.
2011b. TBR1 directly represses Fezf2 to control the laminar origin and
development of the corticospinal tract. \emph{Proc Natl Acad Sci USA}
\textbf{108}: 3041--3046.

Hevner RF, Shi LM, Justice N, Hsueh TP, Sheng M, Smiga S, Bulfone A,
Goffinet AM, Campagnoni AT, Rubenstein J. 2001. Tbr1 regulates
differentiation of the preplate and layer 6. \emph{Neuron} \textbf{29}:
353--366.

Hormozdiari F, Penn O, Borenstein E, Eichler EE. 2015. The discovery of
integrated gene networks for autism and related disorders. \emph{Genome
Res} \textbf{25}: 142--154.

Huang T-N, Chuang H-C, Chou W-H, Chen C-Y, Wang H-F, Chou S-J, Hsueh
Y-P. 2014. Tbr1 haploinsufficiency impairs amygdalar axonal projections
and results in cognitive abnormality. \emph{Nat Neurosci} \textbf{17}:
240--247.

Iossifov I, Levy D, Allen J, Ye K, Ronemus M, Lee Y-H, Yamrom B, Wigler
M. 2015. Low load for disruptive mutations in autism genes and their
biased transmission. \emph{Proc Natl Acad Sci USA} \textbf{112}:
E5600--7.

Iossifov I, O'Roak BJ, Sanders SJ, Ronemus M, Krumm N, Levy D, Stessman
HA, Witherspoon KT, Vives L, Patterson KE, et al. 2014. The contribution
of de novo coding mutations to autism spectrum disorder. \emph{Nature}
\textbf{515}: 216--221.

Iossifov I, Ronemus M, Levy D, Wang Z, Hakker I, Rosenbaum J, Yamrom B,
Lee Y-H, Narzisi G, Leotta A, et al. 2012. De novo gene disruptions in
children on the autistic spectrum. \emph{Neuron} \textbf{74}: 285--299.

Jolma A, Yan J, Whitington T, Toivonen J, Nitta KR, Rastas P, Morgunova
E, Enge M, Taipale M, Wei G, et al. 2013. DNA-binding specificities of
human transcription factors. \emph{Cell} \textbf{152}: 327--339.

Kent WJ. 2002. BLAT-\/-the BLAST-like alignment tool. \emph{Genome Res}
\textbf{12}: 656--664.

Kong A, Frigge ML, Masson G, Besenbacher S, Sulem P, Magnusson G,
Gudjonsson SA, Sigurdsson A, Jonasdottir A, Jonasdottir A, et al. 2012.
Rate of de novo mutations and the importance of father's age to disease
risk. \emph{Nature} \textbf{488}: 471--475.

Krumm N, O'Roak BJ, Shendure J, Eichler EE. 2014. A de novo convergence
of autism genetics and molecular neuroscience. \emph{Trends Neurosci}
\textbf{37}: 95--105.

Leone DP, Srinivasan K, Chen B, Alcamo E, McConnell SK. 2008. The
determination of projection neuron identity in the developing cerebral
cortex. \emph{Curr Opin Neurobiol} \textbf{18}: 28--35.

Li H, Durbin R. 2009. Fast and accurate short read alignment with
Burrows-Wheeler transform. \emph{Bioinformatics} \textbf{25}:
1754--1760.

Li J, Shi M, Ma Z, Zhao S, Euskirchen G, Ziskin J, Urban A, Hallmayer J,
Snyder M. 2014. Integrated systems analysis reveals a molecular network
underlying autism spectrum disorders. \emph{Mol Syst Biol} \textbf{10}:
774.

Li Q, Brown JB, Huang H, Bickel PJ. 2011. Measuring reproducibility of
high-throughput experiments. \emph{The Annals of Applied Statistics}
\textbf{5}: 1752--1779.

Long JE, Garel S, Depew MJ, Tobet S, Rubenstein JLR. 2003. DLX5
regulates development of peripheral and central components of the
olfactory system. \emph{J Neurosci} \textbf{23}: 568--578.

MacArthur DG, Tyler-Smith C. 2010. Loss-of-function variants in the
genomes of healthy humans. \emph{Hum Mol Genet} \textbf{19}: R125--30.

Machanick P, Bailey TL. 2011. MEME-ChIP: motif analysis of large DNA
datasets. \emph{Bioinformatics} \textbf{27}: 1696--1697.

McConnell SK. 1991. The Generation of Neuronal Diversity in the Central
Nervous System. \emph{Annual Review of Neuroscience} \textbf{14}:
269--300.

McKenna WL, Betancourt J, Larkin KA, Abrams B, Guo C, Rubenstein JLR,
Chen B. 2011a. Tbr1 and Fezf2 regulate alternate corticofugal neuronal
identities during neocortical development. \emph{J Neurosci}
\textbf{31}: 549--564.

McKenna WL, Betancourt J, Larkin KA, Abrams B, Guo C, Rubenstein JLR,
Chen B. 2011b. Tbr1 and Fezf2 regulate alternate corticofugal neuronal
identities during neocortical development. \emph{J Neurosci}
\textbf{31}: 549--564.

McKenna WL, Ortiz-Londono CF, Mathew TK, Hoang K, Katzman S, Chen B.
2015. Mutual regulation between Satb2 and Fezf2 promotes subcerebral
projection neuron identity in the developing cerebral cortex. \emph{Proc
Natl Acad Sci USA} \textbf{112}: 11702--11707.

McLean CY, Bristor D, Hiller M, Clarke SL, Schaar BT, Lowe CB, Wenger
AM, Bejerano G. 2010. GREAT improves functional interpretation of
cis-regulatory regions. \emph{Nat Biotechnol} \textbf{28}: 495--501.

Molyneaux BJ, Arlotta P, Menezes JRL, Macklis JD. 2007. Neuronal subtype
specification in the cerebral cortex. \emph{Nat Rev Neurosci}
\textbf{8}: 427--437.

Neale BM, Kou Y, Liu L, Ma'ayan A, Samocha KE, Sabo A, Lin C-F, Stevens
C, Wang L-S, Makarov V, et al. 2012. Patterns and rates of exonic de
novo mutations in autism spectrum disorders. \emph{Nature} \textbf{485}:
242--245.

Notwell JH, Chung T, Heavner W, Bejerano G. 2015. A family of
transposable elements co-opted into developmental enhancers in the mouse
neocortex. \emph{Nat Commun} \textbf{6}: 6644.

O'Roak BJ, Deriziotis P, Lee C, Vives L, Schwartz JJ, Girirajan S,
Karakoc E, MacKenzie AP, Ng SB, Baker C, et al. 2011. Exome sequencing
in sporadic autism spectrum disorders identifies severe de novo
mutations. \emph{Nat Genet} \textbf{43}: 585--589.

O'Roak BJ, Vives L, Girirajan S, Karakoc E, Krumm N, Coe BP, Levy R, Ko
A, Lee C, Smith JD, et al. 2012. Sporadic autism exomes reveal a highly
interconnected protein network of de novo mutations. \emph{Nature}
\textbf{485}: 246--250.

Parikshak NN, Luo R, Zhang A, Won H, Lowe JK, Chandran V, Horvath S,
Geschwind DH. 2013. Integrative functional genomic analyses implicate
specific molecular pathways and circuits in autism. \emph{Cell}
\textbf{155}: 1008--1021.

Petrovski S, Wang Q, Heinzen EL, Allen AS, Goldstein DB. 2013. Genic
intolerance to functional variation and the interpretation of personal
genomes. \emph{PLoS Genet} \textbf{9}: e1003709.

Pfaffl MW. 2001. A new mathematical model for relative quantification in
real-time RT-PCR. \emph{Nucleic Acids Res} \textbf{29}: e45.

Ritchie ME, Phipson B, Wu D, Hu Y, Law CW, Shi W, Smyth GK. 2015. limma
powers differential expression analyses for RNA-sequencing and
microarray studies. \emph{Nucleic Acids Res} \textbf{43}: e47--e47.

Rosenbloom KR, Armstrong J, Barber GP, Casper J, Clawson H, Diekhans M,
Dreszer TR, Fujita PA, Guruvadoo L, Haeussler M, et al. 2015. The UCSC
Genome Browser database: 2015 update. \emph{Nucleic Acids Res}
\textbf{43}: D670--81.

Rozen S, Skaletsky H. 2000. Primer3 on the WWW for general users and for
biologist programmers. \emph{Methods Mol Biol} \textbf{132}: 365--386.

Schneider CA, Rasband WS, Eliceiri KW. 2012. NIH Image to ImageJ: 25
years of image analysis. \emph{Nat Methods} \textbf{9}: 671--675.

Sulem P, Helgason H, Oddson A, Stefansson H, Gudjonsson SA, Zink F,
Hjartarson E, Sigurdsson GT, Jonasdottir A, Jonasdottir A, et al. 2015.
Identification of a large set of rare complete human knockouts.
\emph{Nat Genet} \textbf{47}: 448--452.

The ENCODE Project Consortium. 2012. An integrated encyclopedia of DNA
elements in the human genome. \emph{Nature} \textbf{489}: 57--74.

Traylor RN, Dobyns WB, Rosenfeld JA, Wheeler P, Spence JE, Bandholz AM,
Bawle EV, Carmany EP, Powell CM, Hudson B, et al. 2012. Investigation of
TBR1 Hemizygosity: Four Individuals with 2q24 Microdeletions. \emph{Mol
Syndromol} \textbf{3}: 102--112.

Turner TN, Sharma K, Oh EC, Liu YP, Collins RL, Sosa MX, Auer DR, Brand
H, Sanders SJ, Moreno-De-Luca D, et al. 2015. Loss of δ-catenin function
in severe autism. \emph{Nature} \textbf{520}: 51--56.

Uddin M, Tammimies K, Pellecchia G, Alipanahi B, Hu P, Wang Z, Pinto D,
Lau L, Nalpathamkalam T, Marshall CR, et al. 2014. Brain-expressed exons
under purifying selection are enriched for de novo mutations in autism
spectrum disorder. \emph{Nat Genet} \textbf{46}: 742--747.

Wallace VA, Raff MC. 1999. A role for Sonic hedgehog in
axon-to-astrocyte signalling in the rodent optic nerve.
\emph{Development} \textbf{126}: 2901--2909.

Wenger AM, Clarke SL, Notwell JH, Chung T, Tuteja G, Guturu H, Schaar
BT, Bejerano G. 2013. The enhancer landscape during early neocortical
development reveals patterns of dense regulation and co-option.
\emph{PLoS Genet} \textbf{9(8)}: e1003728.

Willsey AJ, Sanders SJ, Li M, Dong S, Tebbenkamp AT, Muhle RA, Reilly
SK, Lin L, Fertuzinhos S, Miller JA, et al. 2013. Coexpression networks
implicate human midfetal deep cortical projection neurons in the
pathogenesis of autism. \emph{Cell} \textbf{155}: 997--1007.

Yu TW, Chahrour MH, Coulter ME, Jiralerspong S, Okamura-Ikeda K, Ataman
B, Schmitz-Abe K, Harmin DA, Adli M, Malik AN, et al. 2013. Using
Whole-Exome Sequencing to Identify Inherited Causes of Autism.
\emph{Neuron} \textbf{77}: 259--273.

\end{document}
